\chapter{Analýza požiadaviek}
\label{requirements-analysis}

V~tejto kapitole budú predstavené funkcionality aplikácie, ktoré bude môcť používateľ využívať. Ide o~popis funkčných požiadaviek systému. V~jednotlivých podkapitolách budú tieto funkcionality špecifikované prostredníctvom \textit{user stories}, pričom v~niektorých prípadoch bude doplnený aj návrh užívateľského rozhrania~(UI).

\section{Autentifikácia}

Každý používateľ bude mať možnosť vytvoriť si vlastný chat o~vlastnom datasete. Na~tento účel však bude potrebné, aby aplikácia rozpoznala o~akého používateľa ide. Preto si definujeme nasledujúce scénare:

\begin{enumerate}
\item Ako \textbf{neprihlásený užívateľ} sa môžem \textbf{registrovať}, aby \textbf{som mohol využívať zbytok funkcionalít aplikácie}.
\item Ako \textbf{prihlásený užívateľ} sa môžem \textbf{odhlásiť}, aby \textbf{iný užívateľ nevyužíval môj účet a mohol využívať vlastný}.
\item Ako \textbf{prihlásený užívateľ} môžem \textbf{nahrať dataset}, aby \textbf{iný užívateľ nevyužíval môj účet}.
\end{enumerate}

Po~príchode na~stránku sa používateľovi zobrazí formulár, ktorý mu umožní prihlásiť sa, prípadne prejsť na~ďalší formulár pre~registráciu. Na~registráciu a~následné prihlásenie bude používateľovi stačiť emailová adresa a~heslo (viď obr.~\ref{csv-oracle-ui-navrh-login}). Po~prihlásení sa bude môcť používateľ odhlásiť kliknutím na~tlačidlo odhlásenia, ktoré sa bude nachádzať vždy v pravom hornom rohu.

Taktiež sa bude v~pravom hornom rohu vždy nachádazať aj odkaz na~stránku profilu používateľa, kde si bude môcť skontrolovať svoj email a~v~prípade potreby ho upraviť. Rovnako bude mať možnosť zmeniť si aj heslo. Hodnota aktuálneho hesla sa však z~bezpečnostných dôvodov zobrazovať nebude.

\begin{figure}[H]\centering
\includegraphics[width=140mm]{img/csv-oracle-ui-navrh-login}
\caption{Návrh prihlasovacej obrazovky.}
\label{csv-oracle-ui-navrh-login}
\end{figure}

\section{Dataset}

Keďže základom aplikácie je práca s~datasetmi, nasledujú používateľské scenáre, ktoré ilustrujú kľúčové funkcionality systému:

\begin{enumerate}
\item Ako \textbf{prihlásený používateľ} môžem:
\begin{itemize}
\item \textbf{nahrať jeden alebo viac CSV súborov},
\item \textbf{voliteľne priložiť aj schému vo~formáte CSVW},
\item \textbf{určiť separátor a~kódovanie súboru},
\end{itemize}
aby \textbf{mohla aplikácia dataset analyzovať, a~aby som si ho mohol nechať vysvetliť}.
\item Ako \textbf{prihlásený používateľ} si môžem \textbf{zobraziť všetky moje datasety}, aby \textbf{som si niektorý z~nich mohol zvoliť}.
\item Ako \textbf{prihlásený používateľ} si môžem \textbf{zvoliť konkrétny dataset}, aby \textbf{som si zobrazil chaty o~zvolenom datasete}.
\item Ako \textbf{prihlásený používateľ} môžem \textbf{odstrániť dataset}, aby \textbf{som sa zbavil datasetov, ktoré už pre~mňa nie sú relevantné}.
\end{enumerate}

Formulár pre~vytvorenie datasetu bude umiestnený na~samostatnej stránke. Na~zobrazovanie existujúcich datasetov, ich výber a~mazanie bude slúžiť iná stránka, na~ktorej budú datasety zobrazené v~centrálnom paneli. Na~pravej strane bude umiestnené tlačidlo vedúce k~stránke s~formulárom pre~nahratie nového datasetu. Po~výbere datasetu sa zobrazí jeho detail spolu s~tlačidlom ,,Choose dataset``, ktoré používateľovi umožní prejsť na~stránku obsahujúcu chaty o~zvolenom datasete (viď obr.~\ref{csv-oracle-ui-navrh-datasets}).

\begin{figure}[H]\centering
\includegraphics[width=140mm]{img/csv-oracle-ui-navrh-datasets}
\caption{Návrh stránky s datasetmi.}
\label{csv-oracle-ui-navrh-datasets}
\end{figure}

\section{Chat}
\label{chat}

Akonáhle užívateľ zvolí dataset, dostane sa na~stránku, kde sa budú nachádzať chaty o~zvolenom datasete. V~tejto časti aplikácie existujú nasledovné scénare:

\begin{enumerate}
\item Ako \textbf{prihlásený používateľ} si môžem \textbf{vytvoriť chat o~zvolenom datasete}, aby \textbf{som sa mohol rozprávať o~zvolenom datasete}.
\item Ako \textbf{prihlásený používateľ} si môžem \textbf{zobraziť chaty o~zvolenom datasete}, aby \textbf{som sa mohol dostať k~požadovanému chatu}.
\item Ako \textbf{prihlásený používateľ} si môžem \textbf{zvoliť chat}, aby:
\begin{itemize}
\item \textbf{som si mohol čítať správy chatu},
\item \textbf{som sa mohol pýtať asistenta na~otázky týkajúce sa datasetu},
\item \textbf{som mohol prezerať DK},
\item \textbf{som mohol požiadať asistenta o~úpravu DK}.
\end{itemize}
\item Ako \textbf{prihlásený používateľ} môžem \textbf{odstrániť chat}, aby \textbf{som sa zbavil chatov, ktoré už pre~mňa nie sú relevantné}.
\end{enumerate}

Formulár pre~vytvorenie chatu bude umiestnený na~samostatnej stránke a~bude obsahovať políčka pre~zadanie názvu a~voliteľného \textit{user view}. Pomocou \textit{user view} môže používateľ špecifikovať ako dataset vníma, ako sa naň pozerá, môže tu špecifikovať na~čo chce daný dataset využiť. Vzorovým príkladom \textit{user view} by mohlo byť napríklad: ,,I want to use this dataset to create a simple dashboard with charts that show how COVID-19 developed over time, e.g. cases, deaths, and recoveries.``

Na~zobrazenie existujúcich chatov, ich výber, mazanie a~samotné chatovanie s~asistentom bude slúžiť iná stránka (viď obr.~\ref{csv-oracle-ui-navrh-chats}). V~ľavej hornej časti tejto stránky sa budú nachádzať tlačidlá ,,Späť``, ktoré vrátia používateľa na~stránku s~datasetmi, a~tlačidlo na~vytvorenie nového chatu nad~zvoleným datasetom. Po~kliknutí naň bude používateľ presmerovaný na~stránku s~formulárom pre~vytvorenie nového chatu. Okrem týchto dvoch tlačidiel sa bude v~ľavej časti stránky zobrazovať zoznam existujúcich chatov nad aktuálne vybraným datasetom. Každý záznam v~zozname bude obsahovať aj tlačidlo na~jeho odstránenie. Po~výbere konkrétneho chatu sa budú v~pravej časti stránky zobrazovať správy tohto chatu. V~spodnej časti stránky sa bude nachádzať políčko a~tlačidlo na~napísanie a~odoslanie novej správy. Taktiež sa tu bude nachádzať tlačidlo na~zobrazenie/skrytie DK. V~prípade, že bude DK zobrazená, priestor pre~chat sa obmedzí na hornú polovicu obrazovky, zatiaľ čo dolnú polovicu zaberie DK (viď obr.~\ref{csv-oracle-ui-navrh-chats}).

\begin{figure}[H]\centering
\includegraphics[width=140mm]{img/csv-oracle-ui-navrh-chats}
\caption{Návrh stránky s chatmi.}
\label{csv-oracle-ui-navrh-chats}
\end{figure}

\begin{figure}[H]\centering
\includegraphics[width=140mm]{img/csv-oracle-ui-navrh-datasets}
\caption{Návrh stránky s chatmi (DK zobrazená). \textbf{TODO}}
\label{csv-oracle-ui-navrh-datasets}
\end{figure}

\subsection{Vizualizácia DK}
\label{vizualizácia-dk}

Vizualizácia by mala ideálne obsahovať:
\begin{itemize}
\item opis datasetu ako celku, teda o~čom je a~čo reprezentuje
\item pre~každú tabuľku:
\begin{itemize}
\item opis tabuľky
\item vzorové riadky, teda reálne hodnoty z~tabuliek
\item pre~každý stĺpec:
\begin{itemize}
\item opis stĺpca, aké hodnoty reprezentuje
\item dôvod prečo v~stĺpci chýbajú hodnoty, ak chýbajú
\item hodnota a~dôvod korelácie, ak hodnota korelácie s~iným stĺpcom prekročí hodnotu~$0,5$
\end{itemize}
\end{itemize}
\end{itemize}

Údaje budú výsledkom analýzy, ktorá prebehne po~nahratí datasetu. Konkrétne budú generované pomocou data profilovania a~promptovania veľkého jazykového modelu (LLM) s~využitím predpripravených promptov. LLM pritom bude mať prístup k~samotným tabuľkám aj k~reportu z~data profilingu. Môže sa stať, že model nebude schopný vygenerovať relevantnú odpoveď a~daný údaj bude preto vo~vizualizácii chýbať. Tento údaj však môže byť doplnený neskôr počas chatovania medzi používateľom a~asistentom.
