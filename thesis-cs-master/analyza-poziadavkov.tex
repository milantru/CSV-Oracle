\chapter{Analýza požiadaviek}
\label{requirements-analysis}

V~tejto kapitole budú predstavené požiadavky na~systém. Osobitne budú vymenované funkčné a~nefunkčné požiadavky.

\section{Funkčné požiadavky}

Táto podkapitola predstavuje funkcionality, ktoré bude môcť používateľ využívať v~aplikácii. Požiadavky budú špecifikované prostredníctvom \textit{user stories}.

\subsection{Dataset}
\label{dataset}

Jedným zo~základných pilierov aplikácie je práca s~datasetmi. Preto nasledujú používateľské scenáre, ktoré ilustrujú funkcionality systému týkajúce sa datasetov:

\begin{enumerate}
\item Ako \textbf{prihlásený používateľ} môžem:
\begin{itemize}
\item \textbf{nahrať jeden alebo viac CSV súborov},
\item \textbf{voliteľne priložiť aj schému vo~formáte CSVW},
\item \textbf{určiť separátor a~kódovanie súboru},
\end{itemize}
aby \textbf{mohla aplikácia dataset analyzovať, a~aby som si ho mohol nechať vysvetliť}.

\item Ako \textbf{prihlásený používateľ} si môžem \textbf{zobraziť všetky moje nahrané datasety}, aby \textbf{som si niektorý z~nich mohol zvoliť, a~aby som mohol vidieť jeho stav analýzy, teda či už je analýza hotová alebo~sa stále analyzuje}.

\item Ako \textbf{prihlásený používateľ} si môžem \textbf{zvoliť konkrétny dataset}, aby \textbf{som si zobrazil chaty o~zvolenom datasete}.

\item Ako \textbf{prihlásený používateľ} môžem \textbf{odstrániť dataset}, aby \textbf{som sa zbavil datasetov, ktoré už pre~mňa nie sú relevantné}.
\end{enumerate}

\subsection{Chat}
\label{chat}

Akonáhle má používateľ zvolený dataset, bude môcť pracovať s~chatmi týkajúcimi sa zvoleného datasetu. Nasledujú scénare týkajúce sa chatov:
\begin{enumerate}
\item Ako \textbf{prihlásený používateľ} si môžem \textbf{vytvoriť chat o~zvolenom datasete}, aby \textbf{som sa mohol rozprávať o~zvolenom datasete}.

\item Ako \textbf{prihlásený používateľ} si môžem \textbf{zobraziť chaty o~zvolenom datasete}, aby \textbf{som sa mohol dostať k~požadovanému chatu}.

\item Ako \textbf{prihlásený používateľ} si môžem \textbf{zvoliť chat}, aby:
\begin{itemize}
\item \textbf{som si mohol čítať správy chatu},
\item \textbf{som sa mohol pýtať asistenta na~otázky týkajúce sa datasetu},
\item \textbf{som mohol prezerať DK},
\item \textbf{som mohol požiadať asistenta o~úpravu DK}.
\end{itemize}

\item Ako \textbf{prihlásený používateľ} môžem \textbf{odstrániť chat}, aby \textbf{som sa zbavil chatov, ktoré už pre~mňa nie sú relevantné}.
\end{enumerate}

\subsection{DK}
\label{dk}

DK by mala ideálne obsahovať:
\begin{itemize}
\item opis datasetu ako celku, teda o~čom je a~čo reprezentuje

\item pre~každú tabuľku:
\begin{itemize}
\item vzorové riadky, teda reálne hodnoty z~tabuliek
\item opis tabuľky
\item charakteristiku entity alebo~skupiny entít reprezentovaných riadkom tabuľky
\item pre~každý stĺpec:
\begin{itemize}

\item opis stĺpca, aké hodnoty reprezentuje

\item dôvod prečo v~stĺpci chýbajú hodnoty, ak chýbajú

\item hodnota a~dôvod korelácie, ak hodnota korelácie s~iným stĺpcom prekročí hodnotu~$0,5$
\end{itemize}
\end{itemize}
\end{itemize}

\section{Nefunkčné požiadavky}

V nasledujúcom texte sú predstavené nefunkčné požiadavky.

\subsection{Autentifikácia a~autorizácia}
\label{auth-n-auth}

Každý používateľ bude mať možnosť vytvoriť si vlastný chat o~vlastnom datasete. Na~tento účel však bude potrebné, aby aplikácia rozpoznala o~akého používateľa ide. Zároveň nie je žiadúce, aby aplikácia umožnila jednému používateľovi prístup k~obsahu patriacemu inému používatelovi. Preto potrebujeme, aby systém umožnil:
\begin{enumerate}
\item neprihlásenému používateľovi registráciu,
\item registrovanému, ale~neprihlásenému, používateľovi prihlásenie,
\item a~prihlásenému používateľovi odhlásenie.
\end{enumerate}

\section{Dátové profilovanie}

V~kapitole~\ref{limitacie-llm} bolo spomenuté, že LLM majú obmedzené schopnosti uvažovania. Ak by sa používateľ opýtal na~otázky využívajúce výpočty, napr.~na~priemernú hodnotu v~určitom stĺpci, asistent by nemusel odpovedať správne. Navyše, ako bolo spomenuté v~\ref{dk}, prítomnosť niektorých údajov v~DK bude podmienená~-- budeme vyžadovať vysvetlenie chýbajúcich hodnôt len ak nejaké chýbajú a~vysvetlenie korelácie len ak jej hodnota presiahne $0,5$.

Jedným zo~spôsobov ako získať potrebné údaje by bolo umožniť asistentovi využívať nástroj na~výpočet priemeru, príp.~iných hodnôt. Ak by však užívateľ chcel viac podobných údajov vyžadujúcich výpočet, muselo by byť pridaných viacero nástrojov. Vysoký počet nástrojov by mohol spôsobovať LLM problémy pri~výbere správneho nástroja~\cite{too-many-tools}. Ďalšou potenciálnou nevýhodou môže byť dĺžka výpočtu niektorých požadovaných údajov. V~prípade, že by sa hodnota vypočítavala pri~každej požiadavke používateľa, mohlo by to negatívne ovplyvniť používateľskú skúsenosť.

Ďalším možným riešením je vykonanie dátového profilovania (,,data profiling``). Ide o~proces analyzovania a~hodnotenia kvality, štruktúry a~obsahu dát. Výsledkom profilovania môžu byť údaje, ako napríklad minimálne a~maximálne hodnoty v~stĺpcoch, priemerné hodnoty, korelácie, najčastejšie sa vyskytujúce hodnoty a~podobne. V~prípade zvolenia tejto možnosti by postačovalo vykonať dátové profilovanie iba raz a~výsledky uložiť. Pri~požiadavke používateľa na~získanie určitej hodnoty alebo~pri~analýze datasetu by bolo možné výsledok jednoducho načítať, bez~potreby pridávania ďalších nástrojov asistentovi a~bez zdržania spôsobeného opakovanými výpočtami. Na~základe uvedených skutočností volíme toto riešenie.

\section{Očakávaný priebeh využitia aplikácie}
\label{ocakavany-priebeh-vyuzitia-aplikacie}

Po~vymenovaní požiadaviek na~systém sa teraz zameriame na~priebeh využitia základnej funkcionality aplikácie. Tento priebeh je nasledovný:
\begin{enumerate}
\item Používateľ sa zaregistruje a~prihlási do~aplikácie.
\item Používateľ nahrá dataset pozostavajúci z~jedného alebo~viacerých CSV súborov, prípadne nahrá aj CSVW schému.
\item Po~nahratí súborov sa vykoná analýza, t.j.:
\begin{enumerate}
\item spustí sa dátové profilovanie, ktorého výsledkom je report,
\item z~reportu sa vezmú potrebné údaje, napr.~hodnoty korelácie, na~vytvorenie promptov (ďalej už len predpripravené prompty),
\item predpripravené prompty sú využité na~promptovanie LLM, ktorý bude mať prístup k~nahraným CSV súborom, reportom, príp.~aj k~CSVW schéme ak bola nahraná,
\item výsledkom sú informácie o~datasete, pomocou ktorých sa vytvorí DK. \end{enumerate}
\item Po~dokončení analýzy si používateľ zvolí dataset a~vytvorí nový chat, alternatívne zvolí existujúci, kde môže klásť otázky ohľadom datasetu a~asistent mu na~nich bude odpovedať.
\item Ak sa pri~chatovaní objavia informácie, ktoré by chcel používateľ zapísať do~DK, môže o~to požiadať asistenta, a~ten DK aktualizuje. \end{enumerate}
