\chapter{Súvisiace práce}

Vo výslednej aplikácii budeme využívať AI (artificial intelligence) agenta, ktorý bude schopný narábať s~rôznymi nástrojmi. Konkrétne pôjde o~\textit{ReAct} agenta, ktorý bol predstavený v~práci \textit{ReAct: Synergizing Reasoning and Acting in Language Models}~\cite{reactpaper}

Práca predstavuje \textit{ReAct}, novú paradigmu, ktorá spája uvažovanie (\textbf{re}asoning) a~konanie (\textbf{act}ing) vo~veľkých jazykových modeloch (LLM) s~cieľom zlepšiť schopnosti riešenia úloh v~rôznych doménach. \textit{ReAct} umožňuje LLM generovať stopy uvažovania a~akcie špecifické pre~danú úlohu prekladaným spôsobom, čo umožňuje vytvárať, udržiavať a~upravovať plány popri interagovaní s~externými nástrojmi. 
Medzi kľúčové príspevky a zistenia patria:

\begin{itemize}
\item \textbf{Vylepšený výkon:} ReAct prekonáva existujúce metódy v~rôznych úlohách.
\item \textbf{Ľudská interpretovateľnosť:} ReAct generuje trajektórie riešenia úloh podobné ľuďom, ktoré sú interpretovateľnejšie a dôveryhodnejšie.
\item \textbf{Učenie z malého množstva príkladov (Few-shot leaning):} ReAct dosahuje vysokú mieru úspešnosti s minimálnym počtom príkladov v kontexte.
\item \textbf{Obmedzenia a budúca práca:} Práca upozorňuje na výzvy, napr. chyby chyby pri uvažovaní a navrhuje možné riešenia.
\end{itemize}

\textit{ReAct} predstavuje významný krok smerom k~integrácii uvažovania a~konania v~rámci veľkých jazykových modelov pre~spoľahlivé rozhodovanie a~úlohy vyžadujúce uvažovanie.
