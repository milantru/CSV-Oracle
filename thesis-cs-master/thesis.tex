%%% Hlavní soubor. Zde se definují základní parametry a odkazuje se na ostatní části. %%%

% Meta-data o práci (je nutno upravit)
\input metadata.tex

% Vygenerujeme metadata ve formátu XMP pro použití balíčkem pdfx
\input xmp.tex

%% Verze pro jednostranný tisk:
% Okraje: levý 40mm, pravý 25mm, horní a dolní 25mm
% (ale pozor, LaTeX si sám přidává 1in)
\documentclass[12pt,a4paper]{report}
\setlength\textwidth{145mm}
\setlength\textheight{247mm}
\setlength\oddsidemargin{15mm}
\setlength\evensidemargin{15mm}
\setlength\topmargin{0mm}
\setlength\headsep{0mm}
\setlength\headheight{0mm}
% \openright zařídí, aby následující text začínal na pravé straně knihy
\let\openright=\clearpage

%% Pokud tiskneme oboustranně:
% \documentclass[12pt,a4paper,twoside,openright]{report}
% \setlength\textwidth{145mm}
% \setlength\textheight{247mm}
% \setlength\oddsidemargin{14.2mm}
% \setlength\evensidemargin{0mm}
% \setlength\topmargin{0mm}
% \setlength\headsep{0mm}
% \setlength\headheight{0mm}
% \let\openright=\cleardoublepage

%% Pokud práci odevzdáváme pouze elektronicky, vypadají lépe symetrické okraje
% \documentclass[12pt,a4paper]{report}
% \setlength\textwidth{145mm}
% \setlength\textheight{247mm}
% \setlength\oddsidemargin{10mm}
% \setlength\evensidemargin{10mm}
% \setlength\topmargin{0mm}
% \setlength\headsep{0mm}
% \setlength\headheight{0mm}
% \let\openright=\clearpage

%% Vytváříme PDF/A-2u
\usepackage[a-2u]{pdfx}

%% Přepneme na českou sazbu a fonty Latin Modern
\usepackage[czech]{babel}
\usepackage{lmodern}

% Pokud nepouživáme LuaTeX, je potřeba ještě nastavit kódování znaků
\usepackage{iftex}
\ifpdftex
\usepackage[utf8]{inputenc}
\usepackage[T1]{fontenc}
\usepackage{textcomp}
\fi

%%% Další užitečné balíčky (jsou součástí běžných distribucí LaTeXu)
\usepackage{amsmath}        % rozšíření pro sazbu matematiky
\usepackage{amsfonts}       % matematické fonty
\usepackage{amsthm}         % sazba vět, definic apod.
\usepackage{bm}             % tučné symboly (příkaz \bm)
\usepackage{booktabs}       % lepší vodorovné linky v tabulkách
\usepackage{caption}        % umožní definovat vlastní popisky plovoucích objektů
\usepackage{csquotes}       % uvozovky závislé na jazyku
\usepackage{dcolumn}        % vylepšené zarovnání sloupců tabulek
\usepackage{floatrow}       % umožní definovat vlastní typy plovoucích objektů
\usepackage{graphicx}       % vkládání obrázků
\usepackage{icomma}         % inteligetní čárka v matematickém módu
\usepackage{indentfirst}    % zavede odsazení 1. odstavce kapitoly
\usepackage[nopatch=item]{microtype}  % mikrotypografická rozšíření
\usepackage{paralist}       % lepší enumerate a itemize
\usepackage[nottoc]{tocbibind} % zajistí přidání seznamu literatury,
                            % obrázků a tabulek do obsahu
\usepackage{xcolor}         % barevná sazba

% Balíček hyperref, kterým jdou vyrábět klikací odkazy v PDF,
% ale hlavně ho používáme k uložení metadat do PDF (včetně obsahu).
% Většinu nastavítek přednastaví balíček pdfx.
\hypersetup{unicode}
\hypersetup{breaklinks=true}

% Balíčky pro sazbu informatických prací
\usepackage{algpseudocode}  % součást balíčku algorithmicx
\usepackage[Algoritmus]{algorithm}
\usepackage{fancyvrb}       % vylepšené prostředí verbatim
\usepackage{listings}       % zvýrazňování syntaxe zdrojových textů

% Cleveref může zjednodušit odkazování, ale jeho užitečnost pro češtinu
% je minimalní, protože nezvládá skloňování.
% \usepackage{cleveref}

% Formátování bibliografie (odkazů na literaturu)
% Detailní nastavení můžete upravit v souboru macros.tex.
%
% POZOR: Zvyklosti různých oborů a kateder se liší. Konzultujte se svým
% vedoucím, jaký formát citací je pro vaši práci vhodný!
%
% Základní formát podle normy ISO 690 s číslovanými odkazy
\usepackage[natbib,style=iso-numeric,sorting=none]{biblatex}
% ISO 690 s alfanumerickými odkazy (zkratky jmen autorů)
%\usepackage[natbib,style=iso-alphabetic]{biblatex}
% ISO 690 s citacemi tvaru Autor (rok)
%\usepackage[natbib,style=iso-authoryear]{biblatex}
%
% V některých oborech je běžnější obyčejný formát s číslovanými odkazy
% (sorting=none říká, že se bibliografie má řadit podle pořadí citací):
%\usepackage[natbib,style=numeric,sorting=none]{biblatex}
% Číslované odkazy, navíc se [1,2,3,4,5] komprimuje na [1-5]
%\usepackage[natbib,style=numeric-comp,sorting=none]{biblatex}
% Obyčejný formát s alfanumerickými odkazy:
%\usepackage[natbib,style=alphabetic]{biblatex}

% Z tohoto souboru se načítají položky bibliografie
\addbibresource{literatura.bib}

% Definice různých užitečných maker (viz popis uvnitř souboru)
\input macros.tex

%%% Titulní strana a různé povinné informační strany
\begin{document}
%%% Titulní strana práce a další povinné informační strany

%%% Nápisy na přední straně desek
%%% Pokud je práce ve slovenštině, desky mají být česky.

% Desky obvykle nesázíme, ale pokud je chcete přidat, změnte \iffalse na \iftrue
\iffalse

\pagestyle{empty}
\hypersetup{pageanchor=false}
\begin{center}

\large
Univerzita Karlova

\medskip

Matematicko-fyzikální fakulta

\vfill

{\huge\bf\ThesisTypeTitle}

\vfill

{\huge\bf\ThesisTitle\par}

\vfill
\vfill

\hbox to \hsize{\YearSubmitted\hfil \ThesisAuthor}

\end{center}

\newpage\openright
\setcounter{page}{1}

\fi

%%% Titulní strana práce
%%% Pokud je práce ve slovenštině, tato strana zůstává česky.

\pagestyle{empty}
\hypersetup{pageanchor=false}

\begin{center}

\centerline{\mbox{\includegraphics[width=166mm]{img/logo-cs.pdf}}}

\vspace{-8mm}
\vfill

{\bf\Large\ThesisTypeTitle}

\vfill

{\LARGE\ThesisAuthor}

\vspace{15mm}

{\LARGE\bfseries\ThesisTitle\par}

\vfill

\Department

\vfill

{
\centerline{\vbox{\halign{\hbox to 0.45\hsize{\hfil #}&\hskip 0.5em\parbox[t]{0.45\hsize}{\raggedright #}\cr
Vedoucí \ThesisTypeGenitive{} práce:&\Supervisor \cr
\ifx\ThesisType\TypeRig\else
\noalign{\vspace{2mm}}
Studijní program:&\StudyProgramme \cr
\fi
}}}}

\vfill

Praha \YearSubmitted

\end{center}

\newpage

%%% Strana s čestným prohlášením k práci
%%% Pokud je práce ve slovenštině, tato strana zůstává česky.

\openright
\hypersetup{pageanchor=true}
\vglue 0pt plus 1fill

\noindent
Prohlašuji, že jsem tuto \ThesisTypeAccusative{} práci vypracoval(a) samostatně a výhradně
s~použitím citovaných pramenů, literatury a dalších odborných zdrojů.
Beru na~vědomí, že se na moji práci vztahují práva a povinnosti vyplývající
ze zákona č. 121/2000 Sb., autorského zákona v~platném znění, zejména skutečnost,
že Univerzita Karlova má právo na~uzavření licenční smlouvy o~užití této
práce jako školního díla podle §60 odst. 1 autorského zákona.

\vspace{10mm}

\hbox{\hbox to 0.5\hsize{%
V \hbox to 6em{\dotfill} dne \hbox to 6em{\dotfill}
\hss}\hbox to 0.5\hsize{\dotfill\quad}}
\smallskip
\hbox{\hbox to 0.5\hsize{}\hbox to 0.5\hsize{\hfil Podpis autora\hfil}}

\vspace{20mm}
\newpage

%%% Poděkování

\openright

\noindent
\Dedication

\newpage

%%% Povinná informační strana práce

\openright
{\InfoPageFont

\vtop to 0.5\vsize{
\setlength\parindent{0mm}
\setlength\parskip{5mm}

Název práce:
\ThesisTitle

Autor:
\ThesisAuthor

\DeptType:
\Department

Vedoucí \ThesisTypeGenitive{} práce:
\Supervisor, \SupervisorsDepartment

Abstrakt:
\Abstract

Klíčová slova:
{\def\sep{\unskip, }\ThesisKeywords}

\vfil
}

\vtop to 0.49\vsize{
\setlength\parindent{0mm}
\setlength\parskip{5mm}

Title:
\ThesisTitleEN

Author:
\ThesisAuthor

\DeptTypeEN:
\DepartmentEN

Supervisor:
\Supervisor, \SupervisorsDepartmentEN

Abstract:
\AbstractEN

Keywords:
{\def\sep{\unskip, }\ThesisKeywordsEN}

\vfil
}

}

\newpage

%%% Další stránky budeme číslovat
\pagestyle{plain}


%%% Strana s automaticky generovaným obsahem práce

\tableofcontents

%%% Jednotlivé kapitoly práce jsou pro přehlednost uloženy v samostatných souborech
\chapter*{Úvod}
\addcontentsline{toc}{chapter}{Úvod}

\section{Cieľ práce}

Cieľom diplomovej práce je navrhnúť a implementovať systém pre softvérových inžinierov, ktorý po nahraní CSV datasetu pomôže jeho lepšiemu pochopeniu a zároveň poskytne podporu pri rozhodovaní, či je daný dataset vhodný na ďalšie použitie. Systém umožní zhodnotiť, či dataset obsahuje relevantné údaje a či ho možno použiť na splnenie zámeru softvérového inžiniera (napr. vytvorenie softvéru).

Priebeh (flow) programu:
\begin{itemize}
\item Užívateľ nahrá CSV dataset
\item Pustí sa naň data profiling
\item Z výsledku data profilingu si vezmeme potrebné info.
\item Použijeme predpripravené prompty spolu s výsledkom data profilngu na vympromptovanie ľubovoľného LLM a získame tak info o datasete
\item Info o datasete zobrazíme užívateľovi a umožníme mu pýtať sa ďalej na otázky ohľadom datasetu
\item Pri chatovaní sa môžu objaviť nové informácie o datasete, tie sa majú rozpoznať a zobraziť pri už existujúcich (môžu sa prepísať/zhrnúť, aby spolu dávali zmysel, a aby neboli napr. 2 rovnaké vety)
\end{itemize}

\subsection{Pridaná hodnota}

Prečo využivať OracleCSV, nestačí ChatGPT? Teoreticky áno, stačilo by keby užívateľ využil ChatGPT a dostal by podobný výsledok. Pridaná hodnota OracleCSV spočíta v tom, že v prvej fázy systém pomocou predpripravených promptov získa bohaté info o datasete. Užívateľ si ich nemusí vymýšľať sám a ani nemusí vedieť ako sa robí data profiling, čiže dochádza k úspore času i úsilia. Ďalej užívateľovi umožníme rozprávať sa o dasete, pričom si ńebude musieť robiť poznámky, pretože systém mu ich (dokonca zosuamrizované) spíše zaňho.

\section{High-value datasety}

Otvorené dáta sú XYZ, sú zdarma, môžeme využívať. Existujú datasety, ktoré boli označené, že majú "vysokú hodnotu", volajú sa high-value datasety. Asi využijeme práve nejaký high-value dataset. Plus niekedy teraz v roku 2024 myslím, že vychádza zákon, že štáty EU musia vytvárať také datasety.

\section{Predpoklad anglicky hovoriaceho užívateľa}

Ako zistíme jazyk užívateľa? Predstava: Vykonajú sa predvytvorené prompty, získané info sa zobrazí v angličtine? Jazyk vieme získať z inputu, konkrétne z "Additional info" alebo "User view". Ale "Additional info" nemusí byť v jazyku užívateľa (možno ani view?). Návrhy:
\begin{itemize}
\item Užívateľ ako input zadá jazyk (príp. sa vezme z jazyku stránky ak viacjazyčná, ale to asi nebude, lebo to neni náš cieľ)
\item Prvý sa ozve užívateľ a z toho vydedukujeme jeho jazyk (Ako by užívateľ začal? Čo ak z view sa dá zistiť užívateľova otázka/odpoveď)
\item Začneme defaultne v angličtine (teda aj zistené info promptami bude vypísané v angličtine) a jazyk zmeníme:
\begin{itemize}
\item ak si to užívateľ vyžiada
\item vždy keď detekujeme zmenu jazyka (príde mi čudné ak by to systém stále text prekladal, čo ak by kvôli neustálym prekladom vznikli chyby?)
\end{itemize}
\item Obmedzíme sa len na angličtinu (momentálne je v inštrukciách, že užívateľ hovorí anglicky)
\end{itemize}

\section{Výber knižnice pre data profiling}

Pre data profiling si volim Python, kvôli množstvu knižníc a kvôli možnosti rýchlo a pohodlne napísať potrebný kód. Medzi najpouzivanejsie Python kniznice patria podla \href{https://medium.com/@seckindinc/data-profiling-with-python-36497d3a1261}{tohto blogu}:

\begin{itemize}
\item Great expectations -- umoznuje validaciu a data profiling (vid \href{https://legacy.017.docs.greatexpectations.io/docs/0.15.50/}{tu}), hlavnym aspektom, ktory kniznica zavadza su expectations. Ide o tvrdenia o nasich datach (\href{https://legacy.017.docs.greatexpectations.io/docs/0.15.50/#expectations}{vid viac tu}). Nas by zaujimala funkcionalita automatickeho data profilingu (vid \href{https://legacy.017.docs.greatexpectations.io/docs/0.15.50/#automated-data-profiling}{tu}), ktory automaticky vygeneruje sadu tzv. expectations. My by sme okrem ineho potrebovali zistovat korelacie medzi stlpcami, ale nikde medzi expectations nevidim corelacie (vid \href{https://legacy.017.docs.greatexpectations.io/docs/0.15.50/#automated-data-profiling}{tu}). Sice sa daju aj vlastne expectatiosn vyrobit, ale ide o pracu navyse. Okrem sa zda ze pouzivanie tejto kniznice (oproti iným) ma pre nas use case zbytocne strmu learning curve (podoprene vyrokom v Cons \href{https://paul-fry.medium.com/data-profiling-using-great-expectations-17776f140cdc#59d3}{tu}).

\item \href{https://github.com/lux-org/lux}{Lux} -- ide o knižnicu, ktorá uľahčuje rýchle a jednoduché skúmanie údajov automatizáciou procesu vizualizácie a analýzy údajov. Jednoduchým vypísaním `DataFramu` v Jupyter notebooku Lux odporučí súbor vizualizácií, ktoré zvýrazňujú zaujímavé trendy a vzory v súbore údajov. Knižnica sa svojími vizualizáciami hodí skôr na exploráciu dát. Naším cieľom nie sú vizualizácie, ale získanie štrukturovaných informácií, ktorými môžeme zlepšiť kvalitu promptovania. Zdá sa, že knižnica neposkytuje možnosť vypísať analýzu v json formáte. 

\item \href{https://github.com/capitalone/DataProfiler}{DataProfiler} -- knižnica určená na jednoduchú analýzu údajov, monitorovanie a zisťovanie citlivých údajov. Pomocou nej môžeme vytvoriť štrukturovaný report v jsone, ktorý by obsahoval požadované dáta (napr. koreláciu).

\item \href{https://docs.profiling.ydata.ai/latest/}{YData Profiling} (stary nazov Pandas Profiling) -- popredná knižnica na profilovanie údajov, ktorá automatizuje a štandardizuje vytváranie podrobných správ so štatistikami a vizualizáciami. Pomocou nej vieme podobne ako v prípade knižnice DataProfiler vytvoriť štrukturované dáta s dostatočnými informáciami. Knižnica sa ľahko využíva (píšu aj na stránke), navyše vie vygenerovať aj HTML output.
\end{itemize}

Z analýzy knižníc vyplýva, že na naše účely sa hodia knižnice Data Profiler a YData Profiling. YData Profiling oproti Data Profileru dokáže vygenerovať aj HTML report, ktorý by sme teoreticky mohli užívateľovi ponúknuť na stiahnutie alebo mu ho vizualizovať na separátnej stránke. Pre naše účely si preto vyberáme YData Profiling.

\subsection{Obmedzenia}

Vybrali sme si teda knižnicu YData Profiling. \href{https://docs.profiling.ydata.ai/latest/getting-started/concepts/#data-types}{Podľa dokumentácie} existuje 9 typov (resp. 10 ak rozlišujeme \verb|Date| a \verb|DateTime|). My nebudeme pokrývať všetky typy, napr. Image neuvažujeme. Predpokladáme dáta numerické, textové, kategorické a aj dátumy. Fungovanie s inými typmi sa nevylučuje, ale neboli vyskúšané.

\section{Spracovanie textového stĺpca}

Data profiling nám poskytuje množstvo informáciím Medzi nimi nám pre textové stĺpce poskytuje informáciu aké hodnoty a v akom počte sa v stĺpci vyskytujú. Na základe týchto informácií by mohlo LLM lepšie určiť význam stĺpca. Problémom je, že takáto informácia môže byť teoreticky príliŠ veľká na to, aby sme ju mohli poskytnúť jazykovému modelu (nevojde sa do context window). Avšak existujú rôzne lingvistické metódy, ktoré by nám mohli pomôcť. ChatGPT navrhol tieto možnosti:

\begin{enumerate}
\item Term Frequency–Inverse Document Frequency (TF-IDF)
Explanation: TF-IDF is a statistical measure used to evaluate how important a word is to a document in a collection. In your case, it can be used to find the most characteristic words in your text column, helping to highlight what the column is generally about.
How it helps: By identifying the highest-scoring terms (with high TF-IDF values), you can summarize the most relevant words in the column. You could then take a subset of the top terms and their counts to feed into a language model, making it easier for the model to generate a summary.
\item Topic Modeling (e.g., LDA)
Explanation: Topic modeling algorithms, like Latent Dirichlet Allocation (LDA), can identify topics within a collection of documents. Each “topic” is essentially a group of words that frequently occur together.
How it helps: By extracting a small number of main topics from the column, you can summarize the types of content that are most prevalent. The model would output a set of topics, each with related words, giving you insight into what the column represents. You can further analyze the top few topics to form a meaningful description of the column.
\item Word Embedding Clustering
Explanation: Word embeddings, such as those from Word2Vec, GloVe, or more recent embeddings like BERT, represent words as vectors based on their context in a large text corpus. By clustering these embeddings, you can group similar values together.
How it helps: After clustering similar terms, you can look at representative words from each cluster to get a sense of the main themes. This technique works well when there is a lot of variation in the values, as it helps reduce the data down to clusters that capture the core concepts.
\item Named Entity Recognition (NER)
Explanation: Named Entity Recognition is an NLP technique for identifying entities in text, such as names, organizations, dates, locations, etc.
How it helps: By running NER on your values, you could detect patterns or the types of entities present in the column, which could inform you about its content. For example, if NER reveals that many values are place names or organization names, this could indicate that the column may represent locations or affiliations.
\item Frequent Pattern Mining
Explanation: Techniques like Apriori or FP-Growth algorithms are typically used for association rule learning but can also be applied to text data.
How it helps: These techniques can identify patterns or frequent word combinations within your values. By finding recurring word sequences or frequent co-occurrences, you can highlight core elements in the column that might signify its meaning.
\item Text Summarization with Extractive Methods
Explanation: Extractive summarization techniques (e.g., TextRank) can help in summarizing key parts of a text by extracting the most representative phrases.
How it helps: By running a summarization algorithm on a sample of the values in your column, you can get concise phrases that capture the essence of the column’s content. This can reduce the length and complexity of the data fed into an LLM for further summarization or interpretation.
\item Similarity Measures for Value Grouping (Cosine Similarity, Levenshtein Distance)
Explanation: Using similarity measures, you can group text values based on how similar they are, which is particularly useful for categorizing variations of the same term.
How it helps: Grouped values can be represented by a central or most common value, reducing the number of unique items. This allows you to create a shorter list of representative values to understand what types of information the column holds.
\end{enumerate}

Teoreticky môžeme techniky aj kombinovať, ale keďže toto nie je hlavným cieľom práce, tak sa obmedzíme na využitie iba jednej techniky.

NER by som vyradil, pretože naco je by nám bolo identifikovanie entít, veď to by pre LLM nemalo byť problém. Modely ako napr. BERT by som tiež nevyužíval, lebo nevieme aký jazyk bude v stĺpci. Síce existujú multilinguálne modely, ale pokrývajú úplne vŠetky jazyky? Asi nie, čo?

%%% Fiktivní kapitola s ukázkami sazby

\chapter{Nápověda k~sazbě}

\section{Úprava práce}

Vlastní text práce je uspořádaný hierarchicky do kapitol a podkapitol,
každá kapitola začíná na nové straně. Text je zarovnán do bloku. Nový odstavec
se obvykle odděluje malou vertikální mezerou a odsazením prvního řádku. Grafická
úprava má být v~celém textu jednotná.

Práce se tiskne na bílý papír formátu A4. Okraje musí ponechat dost místa na vazbu:
doporučen je horní, dolní a pravý okraj $25\,\rm mm$, levý okraj $40\,\rm mm$.
Číslují se všechny strany kromě obálky a informačních stran na začátku práce;
první číslovaná strana bývá obvykle ta s~obsahem.

Písmo se doporučuje dvanáctibodové ($12\,\rm pt$) se standardní vzdáleností mezi řádky
(pokud píšete ve Wordu nebo podobném programu, odpovídá tomu řádkování $1,5$; v~\TeX{}u
není potřeba nic přepínat).

Primárně je doporučován jednostranný tisk (příliš tenkou práci lze obtížně svázat).
Delší práce je lepší tisknout oboustranně a přizpůsobit tomu velikosti okrajů:
$40\,\rm mm$ má vždy \emph{vnitřní} okraj. Rub titulního listu zůstává nepotištěný.

Zkratky použité v textu musí být vysvětleny vždy u prvního výskytu zkratky (v~závorce nebo
v poznámce pod čarou, jde-li o složitější vysvětlení pojmu či zkratky). Pokud je zkratek
více, připojuje se seznam použitých zkratek, včetně jejich vysvětlení a/nebo odkazů
na definici.

Delší převzatý text jiného autora je nutné vymezit uvozovkami nebo jinak vyznačit a řádně
citovat.

\section{Jednoduché příklady}

K~různým účelům se hodí různé typy písma.
Pro běžný text používáme vzpřímené patkové písmo.
Chceme-li nějaký pojem zvýraznit (třeba v~okamžiku definice), používáme obvykle
\textit{kurzívu} nebo \textbf{tučné písmo.}
Text matematických vět se obvykle tiskne pro zdůraznění \textsl{skloněným (slanted)} písmem;
není-li k~dispozici, může být zastoupeno \textit{kurzívou.}
Text, který je chápan doslova (například ukázky programů) píšeme \texttt{psacím strojem}.
Důležité je být ve volbě písma konzistentní napříč celou prací.

Čísla v~českém textu obvykle sázíme v~matematickém režimu s~desetinnou čárkou:
%%% Bez \usepackage{icomma}:
% $\pi \doteq 3{,}141\,592\,653\,589$.
%%% S \usepackage{icomma}:
$\pi \doteq 3,141\,592\,653\,589$.
V~matematických textech je často lepší používat desetinnou tečku
(pro lepší odlišení od čárky v~roli oddělovače).
Nestřídejte však obojí.
Numerické výsledky se uvádějí s~přiměřeným počtem desetinných míst.

Mezi číslo a jednotku patří úzká mezera: šířka stránky A4 činí $210\,\rm mm$, což si
pamatuje pouze $5\,\%$ autorů. Pokud ale údaj slouží jako přívlastek, mezeru vynecháváme:
$25\rm mm$ okraj, $95\%$ interval spolehlivosti.

Rozlišujeme různé druhy pomlček:
červeno-černý (krátká pomlčka),
strana 16--22 (střední),
$45-44$ (matematické minus),
a~toto je --- jak se asi dalo čekat --- vložená věta ohraničená dlouhými pomlčkami.

V~českém textu se používají \uv{české} uvozovky, nikoliv ``anglické''.

% V tomto odstavci se vlnka zviditelňuje
{
\def~{{\tt\char126}}
Na některých místech je potřeba zabránit lámání řádku (v~\TeX{}u značíme vlnovkou):
u~předložek (neslabičnych, nebo obecně jednopísmenných), vrchol~$v$, před $k$~kroky,
a~proto, \dots{} obecně kdekoliv, kde by při rozlomení čtenář \uv{ško\-brt\-nul}.
}

\section{Matematické vzorce a výrazy}

Proměnné sázíme kurzívou (to \TeX{} v~matematickém módu dělá sám, ale
nezapomínejte na to v~okolním textu a také si matematický mód zapněte).
Názvy funkcí sázíme vzpřímeně. Tedy například:
$\var(X) = \E X^2 - \bigl(\E X \bigr)^2$.

Zlomky uvnitř odstavce (třeba $\frac{5}{7}$ nebo $\frac{x+y}{2}$) mohou
být příliš stísněné, takže je lepší sázet jednoduché zlomky s~lomítkem:
$5/7$, $(x+y)/2$.

Není předepsáno, jakým písmem označovat jednotlivé druhy matematických objektů
(matice, vektory, atd.), ale značení pro tentýž druh objektu musí být v~celé
práci používáno stejně. Podobně používáte-li více různých typů závorek, je třeba
dělat to v~celé práci konzistentně.

Nechť
\[   % LaTeXová náhrada klasického TeXového $$
\mathbf{X} = \begin{pmatrix}
      \T{\bm x_1} \\
      \vdots \\
      \T{\bm x_n}
      \end{pmatrix}.
\]
Povšimněme si tečky za~maticí. Byť je matematický text vysázen
ve~specifickém prostředí, stále je gramaticky součástí věty a~tudíž je
zapotřebí neopomenout patřičná interpunkční znaménka. Obecně nechceme
sázet vzorce jeden za druhým a raději je propojíme textem.

Výrazy, na které chceme později odkazovat, je vhodné očíslovat:
\begin{equation}\label{eq01:Xmat}
\mathbf{X} = \begin{pmatrix}
      \T{\bm x_1} \\
      \vdots \\
      \T{\bm x_n}
      \end{pmatrix}.
\end{equation}
Výraz \eqref{eq01:Xmat} definuje matici $\mathbf{X}$. Pro lepší čitelnost
a~přehlednost textu je vhodné číslovat pouze ty výrazy, na které se
autor někde v~další části textu odkazuje. To jest, nečíslujte
automaticky všechny výrazy vysázené některým z~matematických
prostředí.

Zarovnání vzorců do několika sloupečků:
\begin{alignat*}{3}
S(t) &= \pr(T > t),    &\qquad t&>0       &\qquad&\text{ (zprava spojitá),}\\
F(t) &= \pr(T \leq t), &\qquad t&>0       &\qquad&\text{ (zprava spojitá).}
\end{alignat*}

Dva vzorce se spojovníkem:
\begin{equation}\label{eq01:FS}
\left.
\begin{aligned}
S(t) &= \pr(T > t) \\[1ex]
F(t) &= \pr(T \leq t)
\end{aligned}
\;	% zde pomůže ručně vynechat trochu místa
\right\}
\quad t>0 \qquad \text{(zprava spojité).}
\end{equation}

Dva centrované nečíslované vzorce:
\begin{gather*}
\bm Y = \mathbf{X}\bm\beta + \bm\varepsilon, \\[1ex]
\mathbf{X} = \begin{pmatrix} 1 & \T{\bm x_1} \\ \vdots & \vdots \\ 1 &
  \T{\bm x_n} \end{pmatrix}.
\end{gather*}
Dva centrované číslované vzorce:
\begin{gather}
\bm Y = \mathbf{X}\bm\beta + \bm\varepsilon, \label{eq02:Y}\\[1ex]
\mathbf{X} = \begin{pmatrix} 1 & \T{\bm x_1} \label{eq03:X}\\ \vdots & \vdots \\ 1 &
  \T{\bm x_n} \end{pmatrix}.
\end{gather}

Definice rozdělená na dva případy:
\[
P_{r-j}=
\begin{cases}
0, & \text{je-li $r-j$ liché},\\
r!\,(-1)^{(r-j)/2}, & \text{je-li $r-j$ sudé}.
\end{cases}
\]
Všimněte si použití interpunkce v této konstrukci. Čárky a tečky se
dávají na místa, kam podle jazykových pravidel patří.

\begin{align}
x& = y_1-y_2+y_3-y_5+y_8-\dots = && \text{z \eqref{eq02:Y}} \nonumber\\
& = y'\circ y^* = && \text{podle \eqref{eq03:X}} \nonumber\\
& = y(0) y' && \text {z Axiomu 1.}
\end{align}


Dva zarovnané vzorce nečíslované (povšimněte si větších závorek, aby se do nich
vešel vyšší vzorec):
\begin{align*}
L(\bm\theta) &= \prod_{i=1}^n f_i(y_i;\,\bm\theta), \\
\ell(\bm\theta) &= \log\bigl\{L(\bm\theta)\bigr\} =
\sum_{i=1}^n \log\bigl\{f_i(y_i;\,\bm\theta)\bigr\}.
\end{align*}
Dva zarovnané vzorce, první číslovaný:
\begin{align}
L(\bm\theta) &= \prod_{i=1}^n f_i(y_i;\,\bm\theta), \label{eq01:L} \\
\ell(\bm\theta) &= \log\bigl\{L(\bm\theta)\bigr\} =
\sum_{i=1}^n \log\bigl\{f_i(y_i;\,\bm\theta)\bigr\}. \nonumber
\end{align}

Vzorec na dva řádky, první řádek zarovnaný vlevo, druhý vpravo, nečíslovaný:
\begin{multline*}
\ell(\mu,\,\sigma^2) = \log\bigl\{L(\mu,\,\sigma^2)\bigr\} =
\sum_{i=1}^n \log\bigl\{f_i(y_i;\,\mu,\,\sigma^2)\bigr\}= \\
  = -\,\frac{n}{2}\,\log(2\pi\sigma^2) \,-\,
\frac{1}{2\sigma^2}\sum_{i=1}^n\,(y_i - \mu)^2.
\end{multline*}

Vzorec na dva řádky, zarovnaný na $=$, číslovaný uprostřed:
\begin{equation}\label{eq01:ell}
\begin{split}
\ell(\mu,\,\sigma^2) &= \log\bigl\{L(\mu,\,\sigma^2)\bigr\} =
\sum_{i=1}^n \log\bigl\{f(y_i;\,\mu,\,\sigma^2)\bigr\}= \\
& = -\,\frac{n}{2}\,\log(2\pi\sigma^2) \,-\,
\frac{1}{2\sigma^2}\sum_{i=1}^n\,(y_i - \mu)^2.
\end{split}
\end{equation}

\section{Definice, věty, důkazy, \dots}

Konstrukce typu definice, věta, důkaz, příklad, \dots je vhodné
odlišit od okolního textu a~případně též číslovat s~možností použití
křížových odkazů. Pro každý typ těchto konstrukcí je vhodné mít
v~souboru s~makry (\texttt{makra.tex}) nadefinované jedno prostředí,
které zajistí jak vizuální odlišení od okolního textu, tak
automatické číslování s~možností křížově odkazovat.

\begin{definice}\label{def01:1}
  Nechť náhodné veličiny $X_1,\dots,X_n$ jsou definovány na témž
  prav\-dě\-po\-dob\-nost\-ním prostoru $(\Omega,\,\mathcal{A},\,\pr)$. Pak
  vektor $\bm X = \T{(X_1,\dots,X_n)}$ nazveme \emph{náhodným
    vektorem}.
\end{definice}

\begin{definice}[náhodný vektor]\label{def01:2}
  Nechť náhodné veličiny $X_1,\dots,X_n$ jsou definovány na témž
  pravděpodobnostním prostoru $(\Omega,\,\mathcal{A},\,\pr)$. Pak
  vektor $\bm X = \T{(X_1,\dots,X_n)}$ nazveme \emph{náhodným
    vektorem}.
\end{definice}
Definice~\ref{def01:1} ukazuje použití prostředí pro sazbu definice
bez titulku, definice~\ref{def01:2} ukazuje použití prostředí pro
sazbu definice s~titulkem.

\begin{veta}\label{veta01:1}
  Náhodný vektor $\bm X$ je měřitelné zobrazení prostoru
  $(\Omega,\,\mathcal{A},\,\pr)$ do $(\R_n,\,\mathcal{B}_n)$.
\end{veta}

\begin{lemma}[\citet{Andel07}, str. 29]\label{veta01:2}
  Náhodný vektor $\bm X$ je měřitelné zobrazení prostoru
  $(\Omega,\,\mathcal{A},\,\pr)$ do $(\R_n,\,\mathcal{B}_n)$.
\end{lemma}
\begin{dukaz}
  Jednotlivé kroky důkazu jsou podrobně popsány v~práci \citet[str.
  29]{Andel07}.
\end{dukaz}
Věta~\ref{veta01:1} ukazuje použití prostředí pro sazbu matematické
věty bez titulku, lemma~\ref{veta01:2} ukazuje použití prostředí pro
sazbu matematické věty s~titulkem. Lemmata byla zavedena v~hlavním
souboru tak, že sdílejí číslování s~větami.

%%% Fiktivní kapitola s ukázkami citací

\chapter{Odkazy na literaturu}

Při zpracování bibliografie (přehledu použitých zdrojů) se řídíme
normou ISO 690 a zvyklostmi oboru. V~\LaTeX{}u nám pomohou balíčky
\textsf{biblatex}, \textsf{biblatex-iso690}.
Zdroje definujeme v~souboru \texttt{literatura.bib} a pak se na ně v~textu
práce odkazujeme pomocí makra \verb|\cite|. Tím vznikne odkaz v~textu
a~odpovídající položka v~seznamu literatury.

V~matematickém textu obvykle odkazy sázíme ve tvaru
\uv{Jméno autora/autorů [číslo odkazu]}, případně
\uv{Jméno autora/autorů (rok vydání)}.
V~českém/slovenském textu je potřeba se navíc vypořádat
s~nutností skloňovat jméno autora, respektive přechylovat jméno
autorky.
K~doplňování jmen se hodí příkazy \verb|\citet|, \verb|\citep|
z~balíčku \textsf{natbib}, ale je třeba mít na paměti, že
produkují referenci se jménem autora/autorů v~prvním pádě a~jména
autorek jsou nepřechýlena.

Jména časopisů lze uvádět zkráceně, ale pouze v~kodifikované podobě.

Při citování je třeba se vyhnout neověřitelným, nedohledatelným a nestálým zdrojům.
Doporučuje se pokud možno necitovat osobní sdělení, náhodně nalezené webové stránky,
poznámky k přednáškám apod. Citování spolehlivých elektronických zdrojů (maji ISSN
nebo DOI) a webových stránek oficiálních instituci je zcela v pořádku. Citujeme-li
elektronické zdroje, je třeba uvést URL, na němž se zdroj nachází, a~datum přístupu
ke zdroji.

\section{Několik ukázek}

Aktuální verzi této šablony najdete v~gitovém repozitáři \cite{ThesisTemplate}.
Také se může hodit prohlédnout si další návody udržované Martinem Marešem
\cite{ThesisWeb}.

Mezi nejvíce citované statistické články patří práce Kaplana a~Meiera a~Coxe
\cite{KaplanMeier58, Cox72}. \citet{Student08} napsal článek o~t-testu.

Prof. Anděl je autorem učebnice matematické statistiky \cite{Andel98}.
Teorii odhadu se věnuje práce \citet{LehmannCasella98}.
V~případě odkazů na specifickou informaci
(definice, důkaz, \dots) uvedenou v~knize bývá užitečné uvést
specificky číslo kapitoly, číslo věty atp. obsahující požadovanou
informaci, např. viz \citet[Věta 4.22]{Andel07}.

Mnoho článků je výsledkem spolupráce celé řady osob. Při odkazování
v~textu na článek se třemi autory obvykle při prvním výskytu uvedeme
plný seznam: \citet*{DempsterLairdRubin77} představili koncept EM
algoritmu. Respektive: Koncept EM algoritmu byl představen v~práci
Dempstera, Lairdové a~Rubina~\cite{DempsterLairdRubin77}. Při každém
dalším výskytu již používáme zkrácenou verzi:
\citet{DempsterLairdRubin77} nabízejí též několik příkladů použití EM
algoritmu. Respektive: Několik příkladů použití EM algoritmu lze
nalézt též v~práci Dempstera a~kol.~\cite{DempsterLairdRubin77}.

U~článku s~více než třemi autory odkazujeme vždy zkrácenou formou:
První výsledky projektu ACCEPT jsou uvedeny v~práci Genbergové a~kol.~\cite{Genberget08}.
V~textu \emph{nenapíšeme:} První výsledky
projektu ACCEPT jsou uvedeny v~práci \citet*{Genberget08}.

%%% Fiktivní kapitola s ukázkami tabulek, obrázků a kódu

\chapter{Tabulky, obrázky, programy}

Používání tabulek a grafů v~odborném textu má některá společná
pravidla a~některá specifická. Tabulky a grafy neuvádíme přímo do
textu, ale umístíme je buď na samostatné stránky nebo na vyhrazené
místo v~horní nebo dolní části běžných stránek. \LaTeX\ se o~umístění
plovoucích grafů a tabulek postará automaticky.

Každý graf a tabulku
očíslujeme a umístíme pod ně legendu. Legenda má popisovat obsah grafu
či tabulky tak podrobně, aby jim čtenář rozuměl bez důkladného
studování textu práce.

Na každou tabulku a graf musí být v~textu odkaz
pomocí jejich čísla. Na příslušném místě textu pak shrneme ty
nejdůležitější závěry, které lze z~tabulky či grafu učinit. Text by
měl být čitelný a srozumitelný i~bez prohlížení tabulek a grafů a
tabulky a grafy by měly být srozumitelné i~bez podrobné četby textu.

Na tabulky a grafy odkazujeme pokud možno nepřímo v~průběhu běžného
toku textu; místo \emph{\uv{Tabulka~\ref{tab03:Nejaka} ukazuje, že
    muži jsou v~průměru o~$9,9\,\rm kg$ těžší než ženy}} raději napíšeme
\emph{\uv{Muži jsou o~$9,9\,\rm kg$ těžší než ženy (viz
    Tabulka~\ref{tab03:Nejaka})}}.

\section{Tabulky}

\begin{table}[b!]

\centering
%%% Tabulka používá následující balíčky:
%%%   - booktabs (\toprule, \midrule, \bottomrule)
%%%   - dcolumn (typ sloupce D: vycentrovaná čísla zarovnaná na
%%%     desetinnou čárku
%%%     Všimněte si, že ve zdrojovém kódu jsou desetinné tečky, ale
%%%     tisknou se čárky.
%%% Dále používáme příkazy \pulrad a \mc definované v makra.tex

\begin{tabular}{l@{\hspace{1.5cm}}D{.}{,}{3.2}D{.}{,}{1.2}D{.}{,}{2.3}}
\toprule
 & \mc{} & \mc{\textbf{Směrod.}} & \mc{} \\
\pulrad{\textbf{Efekt}} & \mc{\pulrad{\textbf{Odhad}}} & \mc{\textbf{chyba}$^a$} &
\mc{\pulrad{\textbf{P-hodnota}}} \\
\midrule
Abs. člen     & -10.01 & 1.01 & \mc{---} \\
Pohlaví (muž) & 9.89   & 5.98 & 0.098 \\
Výška (cm)    & 0.78   & 0.12 & <0.001 \\
\bottomrule
\multicolumn{4}{l}{\footnotesize \textit{Pozn:}
$^a$ Směrodatná chyba odhadu metodou Monte Carlo.}
\end{tabular}

\caption{Maximálně věrohodné odhady v~modelu M.}\label{tab03:Nejaka}

\end{table}

U~\textbf{tabulek} se doporučuje dodržovat následující pravidla:

\begin{itemize} %% nebo compactitem z balíku paralist
\item Vyhýbat se svislým linkám. Silnějšími vodorovnými linkami
  oddělit tabulku od okolního textu včetně legendy, slabšími
  vodorovnými linkami oddělovat záhlaví sloupců od těla tabulky a
  jednotlivé části tabulky mezi sebou. V~\LaTeX u tuto podobu tabulek
  implementuje balík \texttt{booktabs}. Chceme-li výrazněji oddělit
  některé sloupce od jiných, vložíme mezi ně větší mezeru.
\item Neměnit typ, formát a význam obsahu políček v~tomtéž sloupci
  (není dobré do téhož sloupce zapisovat tu průměr, onde procenta).
\item Neopakovat tentýž obsah políček mnohokrát za sebou. Máme-li
  sloupec \textit{Rozptyl}, který v~prvních deseti řádcích obsahuje
  hodnotu $0,5$ a v~druhých deseti řádcích hodnotu $1,5$, pak tento
  sloupec raději zrušíme a vyřešíme to jinak. Například můžeme tabulku
  rozdělit na dvě nebo do ní vložit popisné řádky, které informují
o~nějaké proměnné hodnotě opakující se v~následujícím oddíle tabulky
  (např. \emph{\uv{Rozptyl${}=0,5$}} a níže \emph{\uv{Rozptyl${}=
      1,5$}}).
\item Čísla v~tabulce zarovnávat na desetinnou čárku.
\item V~tabulce je někdy potřebné používat zkratky, které se jinde
nevyskytují. Tyto zkratky můžeme vysvětlit v~legendě nebo
v~poznámkách pod tabulkou. Poznámky pod tabulkou můžeme využít i
k~podrobnějšímu vysvětlení významu  některých sloupců nebo hodnot.
\end{itemize}

\section{Obrázky}

Dodejme ještě několik rad týkajících se obrázků a grafů.

\begin{itemize}
\item Graf by měl být vytvořen ve velikosti, v~níž bude použit
  v~práci. Zmenšení příliš velkého grafu vede ke špatné čitelnosti
  popisků.
\item Osy grafu musí být řádně popsány ve stejném jazyce, v~jakém je
  psána práce (absenci diakritiky lze tolerovat). Kreslíme-li graf
  hmotnosti proti výšce, nenecháme na nich popisky \texttt{ht} a
  \texttt{wt}, ale osy popíšeme \emph{Výška [cm]} a~\emph{Hmotnost
    [kg]}. Kreslíme-li graf funkce $h(x)$, popíšeme osy $x$ a $h(x)$.
  Každá osa musí mít jasně určenou škálu.
\item Chceme-li na dvourozměrném grafu vyznačit velké množství bodů,
  dáme pozor, aby se neslily do jednolité černé tmy. Je-li bodů mnoho,
  zmenšíme velikost symbolu, kterým je vykreslujeme, anebo vybereme
  jen malou část bodů, kterou do grafu zaneseme. Grafy, které obsahují
  tisíce bodů, dělají problémy hlavně v~elektronických dokumentech,
  protože výrazně zvětšují velikost souborů.
\item Budeme-li práci tisknout černobíle, vyhneme se používání barev.
  Čáry roz\-li\-šu\-je\-me typem (plná, tečkovaná, čerchovaná,\ldots), plochy
  dostatečně roz\-díl\-ný\-mi intensitami šedé nebo šrafováním. Význam
  jednotlivých typů čar a~ploch vysvětlíme buď v~textové legendě ke
  grafu anebo v~grafické legendě, která je přímo součástí obrázku.
\item Vyhýbejte se bitmapovým obrázkům o~nízkém rozlišení a zejména
  JPEGům (zuby a kompresní artefakty nevypadají na papíře pěkně).
  Lepší je vytvářet obrázky vektorově a vložit do textu jako PDF.
\end{itemize}

\section{Programy}

Algoritmy, výpisy programů a popis interakce s~programy je vhodné
odlišit od ostatního textu. Pro programy se hodí prostředí \texttt{lstlisting}
z~\LaTeX{}ového balíčku \texttt{listings}, které umí i syntakticky zvýrazňovat
běžné programovací jazyky. Většinou ho chceme obalit prostředím \texttt{listing},
čímž z~něj uděláme plovoucí objekt s~popiskem (viz Program~\ref{helloworld}).

Pro algoritmy zapsané v~pseudokódu můžeme použít prostředí \texttt{algorithmic}
z~balíčku \texttt{algpseudocode}. Plovoucí objekt z~něj uděláme obalením prostředím
\texttt{algorithm}. Příklad najdete v~Algoritmu~\ref{isprime}.

\begin{listing}
\begin{lstlisting}
#include <stdio.h>

int main(void)
{
	printf("Hello, world!\n");
	return 0;
}
\end{lstlisting}
\caption{Můj první program}
\label{helloworld}
\end{listing}

\begin{algorithm}
\begin{algorithmic}[1]  % [1] způsobí, že číslujeme kroky algoritmu
\Function{IsPrime}{$x$}
	\State $r \gets \mbox{rovnoměrně náhodné celé číslo mezi 2 a~$x-1$}$
	\State $z \gets x \bmod r$
	\If{$z>0$}
		\State Vrátíme \textsc{ne} \Comment{Našli jsme dělitele}
	\Else
		\State Vrátíme \textsc{ano} \Comment{Možná se mýlíme}
	\EndIf
\EndFunction
\end{algorithmic}
\caption{Primitivní pravděpodobnostní test prvočíselnosti. Pokud odpoví \textsc{ne},
	číslo~$x$ určitě není prvočíslem. Pokud odpoví \textsc{ano}, nejspíš se mýlí.}
\label{isprime}
\end{algorithm}

Další možností je použití {\LaTeX}o\-vé\-ho balíčku
\texttt{fancyvrb} (fancy verbatim), pomocí něhož je v~souboru \texttt{makra.tex}
nadefinováno prostředí \texttt{code}. Pomocí něho lze vytvořit
např. následující ukázky.

V~základním nastavení dostaneme:

\begin{code}
> mean(x)
[1] 158.90
> objekt$prumer
[1] 158.90
\end{code}
%$
Můžeme si říci o~menší písmo:
\begin{code}[fontsize=\footnotesize]
> mean(x)
[1] 158.90
> objekt$prumer
[1] 158.90
\end{code}
%$
Nebo vypnout rámeček:
\begin{code}[frame=none]
> mean(x)
[1] 158.90
> objekt$prumer
[1] 158.90
\end{code}
%$
Případně si říci o~užší rámeček:
\begin{code}[xrightmargin=20em]
> mean(x)
[1] 158.90
> objekt$prumer
[1] 158.90
\end{code}
%$

\begin{figure}[p]\centering
\includegraphics[width=140mm, height=140mm]{img/ukazka-obr01}
% Příponu není potřeba explicitně uvádět, pdflatex automaticky hledá pdf.
% Rozměry také není nutné uvádět.
\caption{Náhodný výběr z~rozdělení $\mathcal{N}_2(\boldsymbol{0},\,I)$.}
\label{obr03:Nvyber}

\end{figure}

\begin{figure}[p]\centering
\includegraphics[width=140mm, height=140mm]{img/ukazka-obr02}
\caption{Hustoty několika normálních rozdělení.}
\label{obr03:Nhust}
\end{figure}

\begin{figure}[p]\centering
\includegraphics[width=140mm, height=198mm]{img/ukazka-obr03}
\caption{Hustoty několika normálních rozdělení.}
\label{obr03:Nhust:podruhe}

\end{figure}

%%% Fiktivní kapitola s instrukcemi k PDF/A

\chapter{Formát PDF/A}

Opatření rektora č. 13/2017 určuje, že elektronická podoba závěrečných
prací musí být odevzdávána ve formátu PDF/A úrovně 1a nebo 2u. To jsou
profily formátu PDF určující, jaké vlastnosti PDF je povoleno používat,
aby byly dokumenty vhodné k~dlouhodobé archivaci a dalšímu automatickému
zpracování. Dále se budeme zabývat úrovní 2u, kterou sázíme \TeX{}em.

Mezi nejdůležitější požadavky PDF/A-2u patří:

\begin{itemize}

\item Všechny fonty musí být zabudovány uvnitř dokumentu. Nejsou přípustné
odkazy na externí fonty (ani na \uv{systémové}, jako je Helvetica nebo Times).

\item Fonty musí obsahovat tabulku ToUnicode, která definuje převod z~kódování
znaků použitého uvnitř fontu to Unicode. Díky tomu je možné z~dokumentu
spolehlivě extrahovat text.

\item Dokument musí obsahovat metadata ve formátu XMP a je-li barevný,
pak také formální specifikaci barevného prostoru.

\end{itemize}

Tato šablona používá balíček {\tt pdfx,} který umí \LaTeX{} nastavit tak,
aby požadavky PDF/A splňoval. Metadata v~XMP se generují automaticky podle
informací v~souboru {\tt prace.xmpdata} (na vygenerovaný soubor se můžete
podívat v~{\tt pdfa.xmpi}).

Validitu PDF/A můžete zkontrolovat pomocí nástroje VeraPDF, který je
k~dispozici na \url{http://verapdf.org/}.

Pokud soubor nebude validní, mezi obvyklé příčiny patří používání méně
obvyklých fontů (které se vkládají pouze v~bitmapové podobě a/nebo bez
unicodových tabulek) a vkládání obrázků v~PDF, které samy o~sobě standard
PDF/A nesplňují.

Další postřehy o~práci s~PDF/A najdete na \url{http://mj.ucw.cz/vyuka/bc/pdfaq.html}.


\chapter*{Závěr}
\addcontentsline{toc}{chapter}{Závěr}


%%% Seznam použité literatury
%%% Seznam použité literatury (bibliografie)
%%%
%%% Pro vytváření bibliografie používáme biblatex. Ten zpracovává
%%% citace v textu (např. makro \cite{...}) a vyhledává k nim literaturu
%%% v souboru literatura.bib.
%%%
%%% Podívejte se na nastavení biblatexu v souboru thesis.tex.

%%% Vytvoření seznamu literatury. Pozor, pokud jste necitovali ani jednu
%%% položku, seznam se automaticky vynechá.

% Dovolíme položkám trochu vyčuhovat přes pravý okraj.
\def\bibfont{\hfuzz=2pt}

\printbibliography[heading=bibintoc,title=Literatura]

%%% Kdybyste chtěli bibliografii vytvářet ručně (bez biblatexu), lze to udělat
%%% následovně. V takovém případě se řiďte normou ISO 690 a zvyklostmi v oboru.

\begin{thebibliography}{99}

% \bibitem{lamport94}
%   {\sc Lamport,} Leslie.
%   \emph{\LaTeX: A Document Preparation System}.
%   2. vydání.
%   Massachusetts: Addison Wesley, 1994.
%   ISBN 0-201-52983-1.

\bibitem{opendata}
Open data: Co jsou otevřená data?
\url{https://opendata.gov.cz/informace:start#co_jsou_otev\%C5\%99en\%C3\%A1_data}

\bibitem{csvw} CSVW: CSV on the Web \url{https://csvw.org/}

\bibitem{highvaluedatasets}
European Commission: Open data and high-value datasets
\url{https://digital-strategy.ec.europa.eu/en/factpages/open-data-and-high-value-datasets-step-step-access-guide}

\bibitem{popular-frontend-frameworks}
Medium: Top Frameworks You Should Use To Develop Dynamic Web Applications
\url{https://medium.com/@sigmasolveusa/top-frameworks-you-should-use-to-develop-dynamic-web-applications-4481a48a74f8#74be}

\bibitem{rag}
Nvidia: What Is Retrieval-Augmented Generation, aka RAG?
\url{https://blogs.nvidia.com/blog/what-is-retrieval-augmented-generation/}

\bibitem{langchain-vs-llamaindex}
Datacamp: LangChain vs LlamaIndex
\url{https://www.datacamp.com/blog/langchain-vs-llamaindex?dc_referrer=https\%3A\%2F\%2Fwww.google.com\%2F}

\bibitem{index}
Llamaindex: Indexing
\url{https://docs.llamaindex.ai/en/stable/module_guides/indexing/\#concept}

\bibitem{top-backend-languages}
GeeksForGeeks: Top 7 Programming Languages for Backend Web Development
\url{https://www.geeksforgeeks.org/top-7-programming-languages-for-backend-web-development/}

\bibitem{top-data-science-languages}
Datacamp: Top 12 Programming Languages for Data Scientists in 2025
\url{https://www.datacamp.com/blog/top-programming-languages-for-data-scientists-in-2022}

\bibitem{data-profiling-packages}
Substack: Data Profiling with Python
\url{https://seckindinc.substack.com/p/data-profiling-with-python-36497d3a1261?utm_source=publication-search}

\bibitem{great-expectations}
Great Expectations: Documentation
\url{https://legacy.017.docs.greatexpectations.io/docs/0.15.50/}

\bibitem{expectations}
Great Expectations: Expectations
\url{https://legacy.017.docs.greatexpectations.io/docs/0.15.50/#expectations}

\bibitem{learning-curve}
Great Expectations: Cons
\url{https://paul-fry.medium.com/data-profiling-using-great-expectations-17776f140cdc#59d3}

\bibitem{ydata-profiling-datatypes}
YData Profiling: Data types
\url{https://docs.profiling.ydata.ai/latest/getting-started/concepts/#data-types}

\end{thebibliography}

% TODO
Dostupnosť online zdrojov bola overená dňa 29.4.2025.


%%% Obrázky v práci
%%% (pokud jich je malé množství, obvykle není třeba seznam uvádět)
\listoffigures

%%% Tabulky v práci (opět nemusí být nutné uvádět)
%%% U matematických prací může být lepší přemístit seznam tabulek na začátek práce.
\listoftables

%%% Použité zkratky v práci (opět nemusí být nutné uvádět)
%%% U matematických prací může být lepší přemístit seznam zkratek na začátek práce.
\chapwithtoc{Seznam použitých zkratek}

%%% Součástí doktorských prací musí být seznam vlastních publikací
\ifx\ThesisType\TypePhD
\chapwithtoc{Seznam publikací}
\fi

%%% Přílohy k práci, existují-li. Každá příloha musí být alespoň jednou
%%% odkazována z vlastního textu práce. Přílohy se číslují.
%%%
%%% Do tištěné verze se spíše hodí přílohy, které lze číst a prohlížet (dodatečné
%%% tabulky a grafy, různé textové doplňky, ukázky výstupů z počítačových programů,
%%% apod.). Do elektronické verze se hodí přílohy, které budou spíše používány
%%% v elektronické podobě než čteny (zdrojové kódy programů, datové soubory,
%%% interaktivní grafy apod.). Elektronické přílohy se nahrávají do SISu.
%%% Povolené formáty souborů specifikuje opatření rektora č. 72/2017.
%%% Výjimky schvaluje fakultní koordinátor pro zavěrečné práce.
\appendix
\chapter{Přílohy}

\section{První příloha}

\end{document}
