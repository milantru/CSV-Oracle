%%% Seznam použité literatury (bibliografie)
%%%
%%% Pro vytváření bibliografie používáme biblatex. Ten zpracovává
%%% citace v textu (např. makro \cite{...}) a vyhledává k nim literaturu
%%% v souboru literatura.bib.
%%%
%%% Podívejte se na nastavení biblatexu v souboru thesis.tex.

%%% Vytvoření seznamu literatury. Pozor, pokud jste necitovali ani jednu
%%% položku, seznam se automaticky vynechá.

% Dovolíme položkám trochu vyčuhovat přes pravý okraj.
\def\bibfont{\hfuzz=2pt}

\printbibliography[heading=bibintoc,title=Literatura]

%%% Kdybyste chtěli bibliografii vytvářet ručně (bez biblatexu), lze to udělat
%%% následovně. V takovém případě se řiďte normou ISO 690 a zvyklostmi v oboru.

\begin{thebibliography}{99}

% \bibitem{lamport94}
%   {\sc Lamport,} Leslie.
%   \emph{\LaTeX: A Document Preparation System}.
%   2. vydání.
%   Massachusetts: Addison Wesley, 1994.
%   ISBN 0-201-52983-1.

\bibitem{opendata}
Open data: Co jsou otevřená data?
\url{https://opendata.gov.cz/informace:start#co_jsou_otev\%C5\%99en\%C3\%A1_data}

\bibitem{csvw} CSVW: CSV on the Web \url{https://csvw.org/}

\bibitem{highvaluedatasets}
European Commission: Open data and high-value datasets
\url{https://digital-strategy.ec.europa.eu/en/factpages/open-data-and-high-value-datasets-step-step-access-guide}

\bibitem{popular-frontend-frameworks}
Roadmap: Top 7 Frontend Frameworks to Use in 2025
\url{https://roadmap.sh/frontend/frameworks}

\bibitem{rag}
Nvidia: What Is Retrieval-Augmented Generation, aka RAG?
\url{https://blogs.nvidia.com/blog/what-is-retrieval-augmented-generation/}

\bibitem{langchain-vs-llamaindex}
Datacamp: LangChain vs LlamaIndex
\url{https://www.datacamp.com/blog/langchain-vs-llamaindex?dc_referrer=https\%3A\%2F\%2Fwww.google.com\%2F}

\bibitem{index}
Llamaindex: Indexing
\url{https://docs.llamaindex.ai/en/stable/module_guides/indexing/\#concept}

\bibitem{top-backend-languages}
GeeksForGeeks: Top 7 Programming Languages for Backend Web Development
\url{https://www.geeksforgeeks.org/top-7-programming-languages-for-backend-web-development/}

\bibitem{top-data-science-languages}
Datacamp: Top 12 Programming Languages for Data Scientists in 2025
\url{https://www.datacamp.com/blog/top-programming-languages-for-data-scientists-in-2022}

\bibitem{data-profiling-packages}
Substack: Data Profiling with Python
\url{https://seckindinc.substack.com/p/data-profiling-with-python-36497d3a1261?utm_source=publication-search}

\bibitem{great-expectations}
Great Expectations: Documentation
\url{https://legacy.017.docs.greatexpectations.io/docs/0.15.50/}

\bibitem{expectations}
Great Expectations: Expectations
\url{https://legacy.017.docs.greatexpectations.io/docs/0.15.50/#expectations}

\bibitem{learning-curve}
Great Expectations: Cons
\url{https://paul-fry.medium.com/data-profiling-using-great-expectations-17776f140cdc#59d3}

\bibitem{ydata-profiling-datatypes}
YData Profiling: Data types
\url{https://docs.profiling.ydata.ai/latest/getting-started/concepts/#data-types}

\bibitem{llm-limitations}
Learn Prompting: What are the Limitations of LLMs?
\url{https://learnprompting.org/docs/basics/pitfalls}

\bibitem{data-labeling-llm}
Medium: Automatic Data Labeling
\url{https://medium.com/cyabra/automatic-data-labeling-an-advanced-llm-powered-approach-e2a6b941c409}

\bibitem{too-many-tools}
HumanLoop: LLM Evals Done Right
\url{https://humanloop.com/blog/LLM-eval-done-right}
\end{thebibliography}

% TODO
Dostupnosť online zdrojov bola overená dňa 29.4.2025.
