\chapter{Dátové zdroje}

Výsledná aplikácia bude pracovať s~CSV súbormi, teda tabuľkami obsahujúcimi hodnoty oddelené čiarkou. Pre~získanie datasetov využijeme \textit{Open Data}.

\section{Open data}

\textit{Open Data}~\cite{opendata}, alebo otvorené dáta, sú dáta zverejňované spôsobom umožňujúcim diaľkový prístup v~otvorenom a~strojovo čitateľnom formáte, ktorých spôsob ani~účel následného použitia nie je obmedzený. Dáta sú evidované v~Národnom katalógu otvorených dát. Otvorené dáta musia byť predovšetkým:

\begin{enumerate}
\item Prístupné ako dátové súbory v~strojovo čitateľnom a~otvorenom formáte s~úplným a~aktuálnym obsahom databázy alebo~agregovanou štatistikou
\item Opatrené neobmedzujúcimi podmienkami použitia
\item Evidované v~Národnom katalógu otvorených dát~(NKOD) ako priame odkazy na dátové súbory
\item Opatrené dokumentáciou
\item Dostupné na~stiahnutie bez~technických prekážok (registrácia, obmedzenie počtu prístupov, CAPTCHA a~pod.)
\item Pripravené s~cieľom čo najjednoduchšieho strojového spracovania programátormi a~pod.
\item Opatrené kontaktom na~kurátora pre~spätnú väzbu (chyby, žiadosť o~rozšírenie a~pod.)
\end{enumerate}

\section{High-value datasety}

Jeden z datasetov, ktoré v práci budeme využívať patrí medzi \textit{high-value datasety}\cite{highvaluedatasets}, alebo~datasety s~vysokou pridanou hodnotou. Ide o datasety, ktoré sú definované ako údaje, ktorých opätovné použitie prináša významné prínosy pre~spoločnosť a~hospodárstvo. Typicky zahŕňajú geopriestorové údaje, údaje z~pozorovania Zeme a~o~životnom prostredí, meteorologické údaje, štatistické údaje, údaje o~spoločnostiach a~ich vlastníckych vzťahoch, ako aj údaje o~mobilite. Cieľom ich sprístupnenia je výrazne znížiť prekážky vstupu na~európsky trh založený na~údajoch a~zároveň zvýšiť objem opätovne využívaných dát.
