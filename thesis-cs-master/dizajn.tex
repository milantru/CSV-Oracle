\chapter{Dizajn aplikácie}

Kvôli lepšej organizácií kódu bude aplikácia rozdelená na~viacero častí (viď obr.~\ref{containers}). Konkrétne pôjde o:
\begin{itemize}
\item \textbf{Frontend}\-- Užívateľské rozhranie, ktoré je prezentované používateľovi a~u\-mo\-žní mu interakciu so~systémom.

\item \textbf{Aplikačný server}\-- Prijíma používateľské žiadosti, ktoré prichádzajú z~Frontendu, tie spracuje a~vráti odpoveď. Spracovanie žiadosti môže zahŕňať čítanie dát z~databázy alebo~ich zápis do~databázy. Okrem toho môže spracovanie žiadostí vyžadovať využitie Python skriptov alebo~LLM Servera.

\item \textbf{Databáza}\-- Slúži na~perzistentné uloženie dát ako sú napr.~emaily užívateľov, ich datasety a~chaty so~správami a~DK.

\item \textbf{Python skripty}\-- Skripty plniace rôzne funkcie, napr.~generovanie promptov využívaných v~analýze, práca s~LLM~atď.

\item \textbf{LLM server}\-- Slúži predovšetkým na~generovanie odpovede asistenta.

\item \textbf{Úložisko kolekcií}\-- Slúži na~uloženie zdrojov informácií z~používateľom nahraných dát.
\end{itemize}

\begin{figure}[H]\centering
\includegraphics[height=100mm]{img/containers}
\caption{Návrh architektúry systému.}
\label{containers}
\end{figure}

\section{Návrh obrazoviek}

Podkapitola predstavuje návrhy jednotlivých obrazoviek, resp.~stránok. Podrobnejšie opisy stránok obsahujúcich iba formulár budú vzhľadom na~ich jednoduchosť vynechané.

\subsection{Registrácia, prihlásenie a~odhlásenie}

V~časti~\ref{auth-n-auth} je špecifikované, že systém má používateľovi umožniť registráciu, prihlásenie a odhlásenie. Po~príchode na~úvodnú stránku sa preto používateľovi zobrazí formulár, ktorý mu umožní prihlásiť sa, prípadne prejsť na~ďalší formulár pre~registráciu. Na~registráciu a~následné prihlásenie bude používateľovi stačiť emailová adresa a~heslo (viď obr.~\ref{csv-oracle-ui-navrh-login}). Po~prihlásení sa bude môcť používateľ odhlásiť kliknutím na~tlačidlo odhlásenia, ktoré sa bude nachádzať vždy v~pravom hornom rohu.

Taktiež sa bude v~pravom hornom rohu vždy nachádazať aj odkaz na~stránku profilu používateľa, kde si bude môcť skontrolovať svoj email a~v~prípade potreby ho upraviť. Rovnako bude mať možnosť zmeniť si aj heslo. Hodnota aktuálneho hesla sa však z~bezpečnostných dôvodov zobrazovať nebude.

\begin{figure}[H]\centering
\includegraphics[width=140mm]{img/csv-oracle-ui-navrh-login}
\caption{Návrh prihlasovacej obrazovky.}
\label{csv-oracle-ui-navrh-login}
\end{figure}

\subsection{Datasety}

V~časti~\ref{dataset} bolo špecifikované, že používateľ bude môcť nahrať dataset s~ďalšími voliteľnými údajmi, že si bude môcť zobraziť všetky svoje nahrané datasety, zvoliť si nejaký z~nich, prípadne nejaký z~nich odstrániť. Preto bude existovať formulár pre~vytvorenie datasetu a~bude umiestnený na~samostatnej stránke. Na~zobrazovanie existujúcich datasetov, ich výber a~mazanie bude slúžiť iná stránka, na~ktorej budú datasety zobrazené v~centrálnom paneli. Na~pravej strane bude umiestnené tlačidlo vedúce k~stránke s~formulárom pre~nahratie nového datasetu. Po~výbere datasetu sa zobrazí jeho detail spolu s~tlačidlom ,,Choose dataset``, ktoré používateľovi umožní prejsť na~stránku obsahujúcu chaty o~zvolenom datasete (viď obr.~\ref{csv-oracle-ui-navrh-datasets}).

\begin{figure}[H]\centering
\includegraphics[width=140mm]{img/csv-oracle-ui-navrh-datasets}
\caption{Návrh stránky s datasetmi.}
\label{csv-oracle-ui-navrh-datasets}
\end{figure}

\subsection{Chaty}

V~časti~\ref{chat} bolo špecifikované, že používateľ bude môcť vytvoriť chat o~zvolenom datasete, že si bude môcť nechať zobraziť všetky chaty o~zvolenom datasete, zvoliť a~využívať nejaký z~nich, čo zahŕňa okrem konverzovania s~asistentom i~používanie DK, príp.~nejaký z~chatov odstrániť.

Preto bude existovať formulár pre~vytvorenie chatu a~bude umiestnený na~samostatnej stránke. Formulár bude obsahovať políčka pre~zadanie názvu a~voliteľného \textit{user view}. Pomocou \textit{user view} môže používateľ špecifikovať ako dataset vníma, ako sa naň pozerá, môže tu špecifikovať na~čo chce daný dataset využiť. Vzorovým príkladom \textit{user view} by mohlo byť napríklad: ,,I want to use this dataset to create a simple dashboard with charts that show how COVID-19 developed over time, e.g. cases, deaths, and recoveries.``

Na~zobrazenie existujúcich chatov, ich výber, mazanie a~samotné chatovanie s~asistentom bude slúžiť iná stránka (viď obr.~\ref{csv-oracle-ui-navrh-chats}). V~ľavej hornej časti tejto stránky sa budú nachádzať tlačidlá ,,Späť``, ktoré vrátia používateľa na~stránku s~datasetmi, a~tlačidlo na~vytvorenie nového chatu nad~zvoleným datasetom. Po~kliknutí naň bude používateľ presmerovaný na~stránku s~formulárom pre~vytvorenie nového chatu. Okrem týchto dvoch tlačidiel sa bude v~ľavej časti stránky zobrazovať zoznam existujúcich chatov nad aktuálne vybraným datasetom. Každý záznam v~zozname bude obsahovať aj tlačidlo na~jeho odstránenie. Po~výbere konkrétneho chatu sa budú v~pravej časti stránky zobrazovať správy tohto chatu. V~spodnej časti stránky sa bude nachádzať políčko a~tlačidlo na~napísanie a~odoslanie novej správy. Taktiež sa tu bude nachádzať tlačidlo na~zobrazenie/skrytie DK. V~prípade, že bude DK zobrazená, priestor pre~chat sa obmedzí na hornú polovicu obrazovky, zatiaľ čo dolnú polovicu zaberie DK (viď obr.~\ref{csv-oracle-ui-navrh-chats-dk}).

\begin{figure}[H]\centering
\includegraphics[width=140mm]{img/csv-oracle-ui-navrh-chats}
\caption{Návrh stránky s chatmi.}
\label{csv-oracle-ui-navrh-chats}
\end{figure}

\begin{figure}[H]\centering
\includegraphics[width=140mm]{img/csv-oracle-ui-navrh-chats}
\caption{Návrh stránky s chatmi (DK zobrazená). \textbf{TODO}}
\label{csv-oracle-ui-navrh-chats-dk}
\end{figure}

\section{Výber technológie pre Frontend}

Z~predchádzajúceho textu vyplýva, že v~aplikácii, napríklad pri~chatovaní alebo~úprave DK, bude potrebné dynamicky meniť používateľské rozhranie (UI). Existuje mnoho frameworkov, ktoré by sa na~túto prácu hodilo. Medzi najpopulárnejšie patria~\cite{popular-frontend-frameworks}: \textit{React}, \textit{Vue}, \textit{Angular}, \textit{Svelte} atď. Nepotrebujeme žiadnu špecifickú funkcionalitu, ktorá by bola limitovaná výlučne na~jeden z~uvedených frameworkov. Požadovaný výsledok by bolo možné dosiahnuť použitím ktoréhokoľvek z~nich. Keďže autor má predchádzajúce skúsenosti s~knižnicou \textit{React}, bola zvolená práve táto možnosť.

\subsection{Polling}

Po~nahratí datasetu sa spustí jeho analýza, ktorá je časovo náročná. Používateľ bude môcť sledovať stav datasetu na~stránke s~datasetmi. Frontend bude periodicky získavať informáciu o~stave datasetu. Stav datasetu bude zobrazený a~jeho možné hodnoty sú:
\begin{itemize}
\item Created~-- dataset je vytvorený, ale~ešte sa nezaradil do~radu na~analýzu
\item Queued~-- dataset je zaradený do~radu na~analýzu, ale~ešte sa nezačalo s~jeho analýzou
\item \textit{Processing}~-- dataset sa analyzuje
\item \textit{Processed}~-- dataset je analyzovaný
\item \textit{Failed}~-- analýza datasetu skončila s~chybou
\end{itemize}

\section{Backend}

Kľúčové oblasti, ktoré bude backendová časť systému riešiť sú:
\begin{enumerate}
\item Autentifikácia
\item Analýza datasetu, do~ktorej patrí:
\begin{itemize}
\item Data profiling
\item Promptovanie LLM
\end{itemize}
\item Práca s~databázou, napr.~na~vytvorenie alebo~odstránenie chatu
\end{enumerate}

Backendová časť aplikácie bude rozdelená na~dve časti:
\begin{itemize}
\item Aplikačný server~-- rieši prvý a tretí bod
\item LLM server~-- rieši druhý bod
\end{itemize}

Dôvod rozdelenia si vysvetlíme ďalej v tejto kapitole.

\section{Výber technológií pre aplikačný server}

Na~implementáciu aplikačného servera je vhodné použiť jazyk~C\# v~kombinácii s~frameworkmi ASP.NET Core a~Entity Framework Core. Ide o~štandardnú a často používanú voľbu v~oblasti vývoja moderných webových aplikácií~\cite{top-backend-languages}, najmä v~prípade potreby silnej typovej kontroly a~integrácie s~relačnými databázami. Táto kombinácia poskytuje výhody v~podobe silného typového systému, ktorý prispieva k~znižovaniu behových chýb, ako aj pohodlného ORM (Object Relational Mapping) prístupu k~databáze. Vďaka využitiu Entity Framework Core a~prístupu \textit{Code First} nie je potrebné manuálne vytvárať SQL schémy~-- databázové tabuľky sú generované automaticky na~základe definícií v~kóde.

Z~pohľadu databázovej vrstvy je vzhľadom na~zvolený .NET stack prirodzenou voľbou relačná databáza Microsoft SQL Server, ktorá sa v praxi často využíva spolu s vyššie uvedenými technológiami. Autor má s jazykmi C\# a~s~technológiami ASP.NET Core a~Entity Framework Core predchádzajúce skúsenosti, čo taktiež prispelo k~výberu tohto riešenia.

\section{Výber technológií pre LLM server}

Pre~implementáciu aplikačného servera v~súvislosti s~tretím bodom však dáva zmysel uvažovať o~použití jazyka Python, keďže ide o~jeden z~najrozšírenejších jazykov v~oblasti dátovej analýzy a~umelej inteligencie~\cite{top-data-science-languages}. Vďaka tomu je k~dispozícii široké spektrum knižníc a~nástrojov, ktoré môžu byť v~rámci implementácie efektívne využité.

\subsection{Výber knižnice pre data profiling}

Medzi najpoužívanejšie knižnice v~jazyku Python určené na~data profiling patria~\cite{data-profiling-packages}:

\begin{itemize}
\item \textbf{Great Expectations}~-- umožňuje validáciu a~profilovanie údajov~\cite{great-expectations}. Kľúčovým konceptom knižnice sú tzv.~\textit{expectations}, čo predstavujú deklaratívne tvrdenia o~vlastnostiach údajov~\cite{expectations}. Relevantná je najmä funkcionalita automatizovaného \href{https://legacy.017.docs.greatexpectations.io/docs/0.15.50/#automated-data-profiling}{profilovania údajov}. Hoci automatizované profilovanie generuje sadu expectations, medzi preddefinovanými možnosťami chýba analýza korelácií medzi stĺpcami. Tie však podľa analýzy požiadavkov chceme (viď~\ref{vizualizácia-dk}). Tento údaj je síce možné doplniť prostredníctvom vlastných expectations, ale~to vyžaduje prácu navyše. Okrem toho sa zdá má knižnica v~porovnaní s~inými pomerne strmú krivku učenia~\cite{learning-curve}.

\item \textbf{Lux}~-- knižnica zameraná na~uľahčenie rýchlej a~intuitívnej exploratívnej analýzy údajov prostredníctvom automatizovaných vizualizácií. Po~vypísaní \texttt{DataFrame} v~Jupyter Notebooku poskytne knižnica návrhy vizualizácií, ktoré zvýrazňujú zaujímavé vzory v~dátach. Knižnica je vhodná najmä na~vizuálne skúmanie údajov. Vzhľadom na~to, že naším cieľom je získanie štruktúrovaných výstupov, ktoré môžu zlepšiť kvalitu promptovania, nie je táto knižnica pre~náš prípad vhodná.

\item \textbf{DataProfiler}~-- knižnica určená na~jednoduchú analýzu údajov, monitorovanie a~zisťovanie rôznych údajov. Poskytuje štruktúrovaný výstup vo~formáte JSON, ktorý obsahuje požadované dáta, napr. koreláciu.

\item \textbf{YData Profiling} (predošlý názov \textit{Pandas Profiling})~-- popredná knižnica na~profilovanie údajov, ktorá automatizuje tvorbu detailných správ obsahujúcich štatistiky a~vizualizácie. Rovnako ako v~prípade DataProfileru, aj tu je možné získať štruktúrovaný výstup s~relevantnými informáciami. Knižnica sa vyznačuje jednoduchosťou použitia a~možnosťou generovať výstup aj vo forme HTML reportu.
\end{itemize}

Na~základe uvedenej analýzy vyplýva, že pre naše účely sú najvhodnejšie knižnice \textbf{DataProfiler} a~\textbf{YData Profiling}. YData Profiling má oproti knižnici DataProfiler výhodu v~tom, že okrem JSON výstupu dokáže vygenerovať aj HTML report, ktorý by sa mohol používateľovi ponúknuť na~stiahnutie alebo~mu ho vizualizovať na~stránke. To síce nie je cieľom tejto práce, ale predstavuje to možné rozšírenie do budúcna. Preto sme sa rozhodli využiť knižnicu \textbf{YData Profiling}.

\subsection{Obmedzenie typov knižnice YData Profiling}

Knižnica YData Profiling rozlišuje deväť dátových typov~\cite{ydata-profiling-datatypes}, resp. desať ak rozlišujeme typy \texttt{Date} a~\texttt{DateTime}. V~rámci tejto práce sa obmedzíme len typy numerické, textové, kategorické a~na~dátumy. Typy ako napr. obrázky, neuvažujeme.

\subsection{Výber knižnice pre prácu s LLM a RAG}

Aby mohol asistent odpovedať na~otázky týkajúce sa konkrétneho datasetu, musí mať prístup k informáciám, ktoré o~ňom existujú. Tieto informácie môže získať prostredníctvom prístupu známeho ako \textit{Retrieval-Augmented Generation} (ďalej len RAG). RAG~\cite{rag} predstavuje metódu, ktorá zvyšuje presnosť a spoľahlivosť generatívnych modelov umelej inteligencie tým, že im poskytuje prístup ku~konkrétnym a~relevantným externým zdrojom údajov.

Medzi najpopulárnejšie knižnice v~jazyku Python určené na~prácu s~veľkými jazykovými modelmi (LLM) a~prístupom RAG patria\textit{LlamaIndex} a~\textit{LangChain}. Pre~potreby tejto práce by bolo možné použiť ktorúkoľvek z~nich, prípadne aj ich kombináciu. Autor však nemá predchádzajúce skúsenosti ani s~jednou z~uvedených knižníc, a~preto by využitie kombinovaného riešenia zbytočne zvyšovalo komplexitu projektu.

Z~dôvodu snahy o~zjednodušenie vývoja a~minimalizovanie začiatočnej záťaže bola zvolená len jedna z~možností~-- konkrétne \textit{LlamaIndex}. Tento nástroj sa orientuje predovšetkým na~oblasť efektívneho získavania dát z~rôznych zdrojov (\textit{data retrieval}), čo presne zodpovedá hlavnému prípadu použitia v~rámci tejto aplikácie~\cite{langchain-vs-llamaindex}.

V~prípade, že sú k~dispozícii údaje, napríklad vo~forme CSV súborov ako v~našom prípade, ktoré chceme sprístupniť veľkému jazykovému modelu s~cieľom rozšírenia kontextu pri generovaní odpovedí, je potrebné z~týchto údajov vytvoriť index. Index~\cite{index} je dátová štruktúra, ktorá umožňuje rýchlo získať relevantný kontext pre~LLM na~odpovedanie použivateľského dotazu a~predstavuje základný pilier knižnice \textit{LlamaIndex}.

\section{Výsledná architektúra}

Pôvodným zámerom bolo vytvoriť aplikačný server s~využitím ASP.NET Core, pričom väčšina backendovej logiky mala byť implementovaná v jazyku C\#. V~prípadoch, keď by bolo potrebné interagovať s~veľkým jazykovým modelom (LLM), by C\# kód spúšťal Python skripty, ktoré by pomocou knižnice \textit{LlamaIndex} zabezpečovali komunikáciu s~LLM.

Počas vývoja sa však ukázalo, že prístup volania Python scriptov zo~C\# kódu nie je z~časových dôvodov vhodný pre~niektoré operácie, napr.~generovanie odpovedí v~rámci chatu. Problém spočíva v~tom, že pri~každom spustení Python skriptu dochádza k~opätovnému načítaniu a~inicializácii všetkých potrebných knižníc, čo spôsobuje výrazné spomalenie celej operácie.

Ako riešenie tohto problému bolo navrhnuté vytvorenie samostatného LLM servera, ktorý zabezpečí jednorazové načítanie a~inicializáciu všetkých potrebných knižníc pri~spustení. Vďaka tomu je možné následne obsluhovať jednotlivé požiadavky efektívne, bez zbytočného opakovania úvodných operácií.
