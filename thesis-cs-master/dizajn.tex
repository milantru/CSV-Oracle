\chapter{Dizajn aplikácie}

Kvôli lepšej organizácií kódu bude aplikácia rozdelená na~viacero častí (viď obr.~\ref{containers}). Konkrétne pôjde o:
\begin{itemize}
\item \textbf{Frontend}\-- Užívateľské rozhranie, ktoré je prezentované používateľovi a~u\-mo\-žní mu interakciu so~systémom.

\item \textbf{Aplikačný server}\-- Prijíma používateľské žiadosti, ktoré prichádzajú z~Frontendu, tie spracuje a~vráti odpoveď. Spracovanie žiadosti môže zahŕňať čítanie dát z~databázy alebo~ich zápis do~databázy. Okrem toho môže spracovanie žiadostí vyžadovať využitie Python skriptov alebo~LLM Servera.

\item \textbf{Databáza}\-- Slúži na~perzistentné uloženie dát ako sú napr.~emaily užívateľov, ich datasety a~chaty so~správami a~DK.

\item \textbf{Python skripty}\-- Skripty plniace rôzne funkcie, napr.~generovanie promptov využívaných v~analýze, práca s~LLM~atď.

\item \textbf{LLM server}\-- Slúži predovšetkým na~generovanie odpovede asistenta.

\item \textbf{Úložisko kolekcií}\-- Slúži na~uloženie zdrojov informácií z~používateľom nahraných dát.
\end{itemize}

\begin{figure}[H]\centering
\includegraphics[height=100mm]{img/containers}
\caption{Návrh architektúry systému.}
\label{containers}
\end{figure}

\section{Návrh obrazoviek}

Podkapitola predstavuje návrhy jednotlivých obrazoviek, resp.~stránok. Podrobnejšie opisy stránok obsahujúcich iba formulár budú vzhľadom na~ich jednoduchosť vynechané.

\subsection{Registrácia, prihlásenie a~odhlásenie}

V~časti~\ref{auth-n-auth} je špecifikované, že systém má používateľovi umožniť registráciu, prihlásenie a odhlásenie. Po~príchode na~úvodnú stránku sa preto používateľovi zobrazí formulár, ktorý mu umožní prihlásiť sa, prípadne prejsť na~ďalší formulár pre~registráciu. Na~registráciu a~následné prihlásenie bude používateľovi stačiť emailová adresa a~heslo (viď obr.~\ref{csv-oracle-ui-navrh-login}). Po~prihlásení sa bude môcť používateľ odhlásiť kliknutím na~tlačidlo odhlásenia, ktoré sa bude nachádzať vždy v~pravom hornom rohu.

Taktiež sa bude v~pravom hornom rohu vždy nachádazať aj odkaz na~stránku profilu používateľa, kde si bude môcť skontrolovať svoj email a~v~prípade potreby ho upraviť. Rovnako bude mať možnosť zmeniť si aj heslo. Hodnota aktuálneho hesla sa však z~bezpečnostných dôvodov zobrazovať nebude.

\begin{figure}[H]\centering
\includegraphics[width=140mm]{img/csv-oracle-ui-navrh-login}
\caption{Návrh prihlasovacej obrazovky.}
\label{csv-oracle-ui-navrh-login}
\end{figure}

\subsection{Datasety}
\label{datasety-navrh}

V~časti~\ref{dataset} bolo špecifikované, že používateľ bude môcť nahrať dataset s~ďalšími voliteľnými údajmi, že si bude môcť zobraziť všetky svoje nahrané datasety i~s ich stavmi, zvoliť si nejaký z~nich, prípadne nejaký z~nich odstrániť. Preto bude existovať formulár pre~vytvorenie datasetu a~bude umiestnený na~samostatnej stránke. Na~zobrazovanie existujúcich datasetov, ich výber a~mazanie bude slúžiť iná stránka, na~ktorej budú datasety zobrazené v~centrálnom paneli. V~tejto časti bude môcť používateľ zároveň sledovať stav datasetov, teda či sú už analyzované alebo~sa analyzujú. Na~pravej strane bude umiestnené tlačidlo vedúce k~stránke s~formulárom pre~nahratie nového datasetu. Po~výbere datasetu sa zobrazí jeho detail spolu s~tlačidlom ,,Choose dataset``, ktoré používateľovi umožní prejsť na~stránku obsahujúcu chaty o~zvolenom datasete (viď obr.~\ref{csv-oracle-ui-navrh-datasets}).

\begin{figure}[H]\centering
\includegraphics[width=140mm]{img/csv-oracle-ui-navrh-datasets}
\caption{Návrh stránky s datasetmi.}
\label{csv-oracle-ui-navrh-datasets}
\end{figure}

\subsection{Chaty}

V~časti~\ref{chat} bolo špecifikované, že používateľ bude môcť vytvoriť chat o~zvolenom datasete, že si bude môcť nechať zobraziť všetky chaty o~zvolenom datasete, zvoliť a~využívať nejaký z~nich, čo zahŕňa okrem konverzovania s~asistentom i~používanie DK, príp.~nejaký z~chatov odstrániť.

Preto bude existovať formulár pre~vytvorenie chatu a~bude umiestnený na~samostatnej stránke. Formulár bude obsahovať políčka pre~zadanie názvu a~voliteľného \textit{user view}. Pomocou \textit{user view} môže používateľ špecifikovať ako dataset vníma, ako sa naň pozerá, môže tu špecifikovať na~čo chce daný dataset využiť. Vzorovým príkladom \textit{user view} by mohlo byť napríklad: ,,I want to use this dataset to create a simple dashboard with charts that show how COVID-19 developed over time, e.g. cases, deaths, and recoveries.``

Na~zobrazenie existujúcich chatov, ich výber, mazanie a~samotné chatovanie s~asistentom bude slúžiť iná stránka (viď obr.~\ref{csv-oracle-ui-navrh-chats}). V~ľavej hornej časti tejto stránky sa budú nachádzať tlačidlá ,,Späť``, ktoré vrátia používateľa na~stránku s~datasetmi, a~tlačidlo na~vytvorenie nového chatu nad~zvoleným datasetom. Po~kliknutí naň bude používateľ presmerovaný na~stránku s~formulárom pre~vytvorenie nového chatu. Okrem týchto dvoch tlačidiel sa bude v~ľavej časti stránky zobrazovať zoznam existujúcich chatov nad aktuálne vybraným datasetom. Každý záznam v~zozname bude obsahovať aj tlačidlo na~jeho odstránenie. Po~výbere konkrétneho chatu sa budú v~pravej časti stránky zobrazovať správy tohto chatu. V~spodnej časti stránky sa bude nachádzať políčko a~tlačidlo na~napísanie a~odoslanie novej správy. Taktiež sa tu bude nachádzať tlačidlo na~zobrazenie/skrytie DK. V~prípade, že bude DK zobrazená, priestor pre~chat sa obmedzí na hornú polovicu obrazovky, zatiaľ čo dolnú polovicu zaberie DK (viď obr.~\ref{csv-oracle-ui-navrh-chats-dk}).

\begin{figure}[H]\centering
\includegraphics[width=140mm]{img/csv-oracle-ui-navrh-chats}
\caption{Návrh stránky s chatmi.}
\label{csv-oracle-ui-navrh-chats}
\end{figure}

\begin{figure}[H]\centering
\includegraphics[width=140mm]{img/csv-oracle-ui-navrh-chats}
\caption{Návrh stránky s chatmi (DK zobrazená). \textbf{TODO}}
\label{csv-oracle-ui-navrh-chats-dk}
\end{figure}

\section{Výber technológie pre Frontend}

Z~predchádzajúceho textu vyplýva, že v~aplikácii, napríklad pri~chatovaní alebo~úprave DK, bude potrebné dynamicky meniť používateľské rozhranie (UI). Existuje mnoho frameworkov, ktoré by sa na~túto prácu hodilo. Medzi najpopulárnejšie patria~\cite{popular-frontend-frameworks}: \textit{React}, \textit{Vue}, \textit{Angular}, \textit{Svelte} atď. Nepotrebujeme žiadnu špecifickú funkcionalitu, ktorá by bola limitovaná výlučne na~jeden z~uvedených frameworkov. Požadovaný výsledok by bolo možné dosiahnuť použitím ktoréhokoľvek z~nich. Keďže autor má predchádzajúce skúsenosti s~knižnicou \textit{React}, bola zvolená práve táto možnosť.

\section{Analýza}

V~\ref{ocakavany-priebeh-vyuzitia-aplikacie} bol opísaný očakávaný priebeh využitia aplikácie, ktorého súčasťou bola analýza datasetu. Jej výsledkom je DK. Analýza sa skladá z:
\begin{itemize}
\item dátového profilovania, ktorého výsledkom je report
\item vytvorenia promptov s~využitím reportov
\item promptovania LLM, ktoré bude mať prístup k~CSV súborom, reportom, príp.~aj k~CSVW schéme ak bola nahraná
\end{itemize}

\subsection{Výber knižnice pre~dátové profilovanie}

Medzi najpoužívanejšie knižnice určené na~data profiling patria~\cite{data-profiling-packages}:
\begin{itemize}
\item \textbf{Great Expectations}~-- umožňuje validáciu a~profilovanie údajov~\cite{great-expectations}. Kľúčovým konceptom knižnice sú tzv.~\textit{expectations}, čo predstavujú deklaratívne tvrdenia o~vlastnostiach údajov~\cite{expectations}. Relevantná je najmä funkcionalita automatizovaného profilovania údajov~\cite{automated-data-profiling}. Hoci automatizované profilovanie generuje sadu expectations, medzi preddefinovanými možnosťami chýba analýza korelácií medzi stĺpcami. Tie však podľa analýzy požiadavkov chceme (viď~\ref{vizualizácia-dk}). Tento údaj je síce možné doplniť prostredníctvom vlastných expectations, ale~to vyžaduje prácu navyše. Okrem toho sa zdá má knižnica v~porovnaní s~inými pomerne strmú krivku učenia~\cite{learning-curve}.

\item \textbf{Lux}~-- knižnica zameraná na~uľahčenie rýchlej a~intuitívnej exploratívnej analýzy údajov prostredníctvom automatizovaných vizualizácií. Po~vypísaní \texttt{DataFrame} v~Jupyter Notebooku poskytne knižnica návrhy vizualizácií, ktoré zvýrazňujú zaujímavé vzory v~dátach. Knižnica je vhodná najmä na~vizuálne skúmanie údajov. Vzhľadom na~to, že naším cieľom je získanie štruktúrovaných výstupov, ktoré môžu zlepšiť kvalitu promptovania, nie je táto knižnica pre~náš prípad vhodná.

\item \textbf{DataProfiler}~-- knižnica určená na~jednoduchú analýzu údajov, monitorovanie a~zisťovanie rôznych údajov. Poskytuje štruktúrovaný výstup vo~formáte JSON, ktorý obsahuje požadované dáta, napr.~koreláciu.

\item \textbf{YData Profiling} (predošlý názov \textit{Pandas Profiling})~-- popredná knižnica na~profilovanie údajov, ktorá automatizuje tvorbu detailných správ obsahujúcich štatistiky a~vizualizácie. Rovnako ako v~prípade DataProfileru, aj tu je možné získať štruktúrovaný výstup vo~formáte JSON s~požadovanými údajmi. Knižnica sa vyznačuje jednoduchosťou použitia a~možnosťou generovať výstup aj vo~forme HTML reportu.
\end{itemize}

Na~základe uvedenej analýzy vyplýva, že pre naše účely sú najvhodnejšie knižnice \textbf{DataProfiler} a~\textbf{YData Profiling}. YData Profiling má oproti knižnici DataProfiler výhodu v~tom, že okrem JSON výstupu dokáže vygenerovať aj HTML report, ktorý by sa mohol používateľovi ponúknuť na~stiahnutie alebo~mu ho vizualizovať na~stránke. To síce nie je cieľom tejto práce, ale~predstavuje to možné rozšírenie do~budúcna. Preto sme sa rozhodli využiť knižnicu \textbf{YData Profiling}.

Knižnica YData Profiling rozlišuje deväť dátových typov~\cite{ydata-profiling-datatypes}, resp. desať ak rozlišujeme typy \texttt{Date} a~\texttt{DateTime}. V~rámci tejto práce sa obmedzíme len typy numerické, textové, kategorické a~na~dátumy. Typy ako napr.~obrázky, neuvažujeme.

\subsection{Vytvorenie promptov}
\label{vytvorenie-promptov}

Táto časť predstaví prompty určené na~získanie informácií o~datasete podľa požiadaviek špecifikovaných v~\ref{dk}, ktoré budú ďalej využité v~DK.

\begin{itemize}
\item \textbf{Prompt pre~získanie opisu datasetu ak ho tvorí jeden súbor:}

,,Write description of~the~dataset , and~try to~explain the~context or~domain the~data comes from.``

\item \textbf{Prompt pre~získanie opisu datasetu ak ho tvorí viacero súborov:}

,,Write description of~the~dataset (considering all tables together) , and~try to~explain the~context or~domain the~data comes from.``

\item \textbf{Prompt pre~získanie opisu tabuľky:}

,,Write description of~the~table '{NÁZOV TABUĽKY}', what kind of~data does the~table contain?``

\item \textbf{Prompt pre~získanie charakteristiky entity alebo~skupiny entít reprezentovaných riadkom tabuľky:}

,,Write what kind of~entity or~entities a~row in~the~table '{NÁZOV TABUĽKY}' represent.``

\item \textbf{Prompt pre~získanie opisu stĺpca:}

,,Write description of~the~column '{NÁZOV STĹPCA}' from the~table '{NÁZOV TABUĽKY}'. Focus on what the~column contains or~measures. Respond only with the~description, no additional text.``

\item \textbf{Prompt pre~získanie vysvetlenie chýbajúcich hodnôt:}

,,The column '{NÁZOV STĹPCA}' from the~table '{NÁZOV TABUĽKY}' has {POČET CHÝBAJÚCICH HODNÔT} missing values ({POČET CHÝ\-BA\-JÚ\-CICH HODNÔT V~PERCENTÁCH}\%). Explain why values might be missing in~this column. Respond only with~the~explanation, no additional text.``

\item \textbf{Prompt pre~získanie vysvetlenie korelácie dvoch stĺpcov:}

,,The~columns '{NÁZOV PRVÉHO STĹPCA}' and~'{NÁZOV DRUHÉHO STĹPCA}' from~the~table '{NÁZOV TABUĽKY}' are~correlated (correlation value: {HODNOTA KORELÁCIE}). Try to~explain why this correlation exists. Respond only with~the~explanation, no additional text. Please do not use terms like 'former' or~'latter', refer to~the~columns by their exact names.``
\end{itemize}

Získanie vzorových riadkov si nevyžaduje použitie promptu. Ide o~informáciu, ktorú nie je potrebné vysvetľovať ani~získavať iným komplexným spôsobom. Stačí riadky prečítať buď z~poskytnutých CSV súborov alebo~z~reportu vygenerovaného dátovým profilovaním, keďže knižnica YData Profiling poskytuje aj túto informáciu.

\subsection{Práca s~LLM}

Poslednou časťou analýzy je promptovanie LLM, ktoré by malo mať prístup k~CSV súborom, reportom a~prípadne aj k~CSVW schéme, ak bola nahraná. Tento prístup je potrebný na~to, aby model mohol odpovedať na~predpripravené prompty definované v~predchádzajúcej časti (viď~\ref{vytvorenie-promptov}).

Analýza však nie je jediným miestom, kde je potrebné využiť LLM s~prístupom k~spomínaným dátam. Podľa požiadaviek uvedených v~časti~\ref{chat} má byť používateľ schopný komunikovať s~asistentom, teda s~LLM, o~datasete. Okrem toho z~požiadaviek vyplýva, že asistent by mal mať schopnosť upravovať DK, čo zahŕňa aj využívanie nástrojov.

Pre~lepšiu prehľadnosť môžeme zhrnúť, že vo~fáze analýzy musí mať LLM prístup k~dátam ako sú CSV súbory, reporty a~prípadne aj k~CSVW schéme, ak bola nahraná, aby mohol odpovedať na~otázky týkajúce sa datasetu. Rovnako tak musí mať LLM prístup k~týmto dátam aj vo~fáze komunikácie s~používateľom. V~tejto fáze však musí byť schopný navyše udržiavať stav konverzácie a~využívať nástroje na~úpravu DK.

Aby mohol asistent odpovedať na~otázky vyžadujúce údaje z~CSV súborov, reportov alebo~schémy, potrebujeme mu tieto údaje nejakým spôsobom predať. Naskytujú sa možnosti~\cite{giving-llm-context}:
\begin{enumerate}
\item Fine-tuning~-- Ide o~proces prispôsobenia predtrénovaného modelu na~konkrétne úlohy alebo~použitia~\cite{fine-tuning}. Tento proces upravuje váhy modelu, čo implikuje nutnosť prístupu k~týmto váham. Táto požiadavka preto~obmedzuje výber modelov, keďže nie všetky váhy sú verejne dostupné. Navyše, ak by sme chceli, aby asistent vedel využívať znalosti z~DK podobne ako napríklad z~CSV súborov alebo~reportov, bolo by potrebné model zakaždým po~úprave DK pretrénovať, čo by mohlo byť z~časového hľadiska nepraktické (záleží od~viacerých faktorov, napr.~veľkost modelu, veľkosť DK).

\item In-Context Learning~-- Proces, pri~ktotom sa požadovaná znalosť vloží v~promptoch do~modelu počas inferencie~\cite{in-context-learning}. Pre~naše úcely je táto možnosť vysoko nepraktická vzhľadom na~potenciálnu veľkosť dát. Dáta by sa nemuseli vôjsť do~kontextového okna modelu.

\item Retrieval-Augmented Generation (ďalej len RAG)~-- Metóda zvyšujúca presnosť a~spoľahlivosť generatívnych modelov tým, že im poskytuje prístup ku~konkrétnym a~relevantným externým zdrojom údajov~\cite{rag}. Ide o~techniku podobnú In-Context Learningu. Dáta, v~našom prípade napr.~CSV súbory, sú rozdelené na~menšie časti, z~ktorých sa vytvárajú vektorové reprezentácie. Tieto vektory sú následne uložené vo~vektorovej databáze. Pri~spracovaní používateľského dotazu sa najprv vygeneruje vektorová reprezentácia dotazu, tzv.~query vektor. Následne sa v~databáze vyhľadá určitý počet vektorov, ktoré sú najpodobnejšie query vektoru a~ich prislúchajúce texty sa použijú ako kontext pri~generovaní odpovede na~používateľský dotaz~\cite{vector-dbs-as-context-source}.
\end{enumerate}

Na~základe vyššie uvedených skutočností možno konštatovať, že pre~dané účely je najvhodnejšou technikou práve metóda RAG. Pri~zmene DK bude potrebné vytvoriť nové vektory alebo~aktualizovať existujúce záznamy vo~vektorovej databáze, pričom však nebude nutné pri~každej zmene pretrénovávať celý model a~do~kontextu budú vložené iba relevantné údaje. Napriek uvedenej nevýhode sa táto možnosť javí ako najefektívnejšie riešenie, a~preto si ju volíme.

\subsection{Výber frameworku pre~prácu s~LLM}

Pre~naše účely bude potrebný framework umožňujúci prácu s~LLM tak, aby bolo možné viesť konverzáciu a~zachovávať históriu predchádzajúcej komunikácie. Ako bolo uvedené v~podkapitole~\ref{limitacie-llm}, LLM si predchádzajúce správy bez~externého riadenia nepamätá. Zároveň musí framework podporovať vyhľadávanie údajov pre~účely implementácie techniky~RAG. Ideálne by mal tiež umožňovať abstraktnú prácu s~LLM, aby systém nebol naviazaný na~konkrétny model a~nebol obmedzený jeho používaním.

Najpopulárnejšími frameworkmi, ktoré spĺňajú vyššie uvedené kritériá sú LangChain a LlamaIndex~\cite{top-rag-frameworks}. Obe frameworky možno využívať v jazykoch Python alebo TypeScript. Python je jedným z~najrozšírenejších jazykov v~oblasti dátovej analýzy a~umelej inteligencie~\cite{top-data-science-languages}, to môže implikovať vyššiu podporu či už v oblasti kódu alebo dokumentácie, ale takisto aj vyššie množstvo knižníc a nástrojov v danej oblasti v prípade potreby.

Medzi najpopulárnejšie frameworky, ktoré spĺňajú všetky uvedené kritériá patrí LangChain a~LlamaIndex~\cite{top-rag-frameworks}. Obidva frameworky sú dostupné pre~programovacie jazyky Python a~TypeScript. Python patrí medzi najrozšírenejšie jazyky v~oblasti dátovej analýzy a~umelej inteligencie~\cite{top-data-science-languages}, čo môže naznačovať vyššiu podporu v~oblasti kódu a~dokumentácie, ako aj širšiu dostupnosť knižníc a~nástrojov v~prípade potreby.

Pre~potreby tejto práce by bolo možné použiť ktorýkoľvek z~nich, dokonca aj ich kombináciu. Autor však nemá predchádzajúce skúsenosti ani s~jedným z~uvedených frameworkov, a~preto by využitie kombinovaného riešenia zbytočne zvyšovalo komplexitu projektu. Z~dôvodu snahy o~zjednodušenie vývoja a~minimalizovanie začiatočnej záťaže bola zvolená len jedna z~možností~-- LlamaIndex. Tento nástroj sa orientuje predovšetkým na~oblasť efektívneho získavania dát z~rôznych zdrojov, tzv.~,,data retrieval``, čo presne zodpovedá hlavnému prípadu použitia v~rámci tejto aplikácie~\cite{langchain-vs-llamaindex}.

\subsection{Index a jeho uloženie}

V~prípade, že sú k~dispozícii údaje, napríklad vo~forme CSV súborov, ako v~našom prípade, ktoré chceme sprístupniť veľkému jazykovému modelu s~cieľom rozšírenia kontextu pri~generovaní odpovedí, je potrebné z~týchto údajov vytvoriť index. Index~\cite{index} je dátová štruktúra, ktorá umožňuje rýchlo získať relevantný kontext pre~LLM na~odpovedanie používateľského dotazu a~predstavuje základný pilier knižnice LlamaIndex.

Aby sme nemuseli index vytvárať zakaždým keď bude potrebný, je vhodné si ho niekam uložiť. Vektorové úložiská (,,Vector Stores``) sú kľúčovou súčasťou pre~využívanie techniky RAG. Vektorové úložisko príjima zoznam tzv.~uzlov (,,Nodes``) a~vybuduje z~nich index. Uzol~\cite{documents-and-nodes} predstavuje časť \textit{dokumentu}, či už ide o~textovú časť, obrázok alebo niečo iné. \textit{Dokument} je všeobecný kontajner okolo akéhokoľvek zdroja dát~-- napríklad PDF súboru, CSV súboru, JSON súboru atď. Trieda pre~praću s~vektorovým úložiskom v~LlamaIndexe sa nazýva \texttt{VectorStoreIndex}. Predvolene \texttt{VectorStoreIndex} ukladá všetko do~pamäte~\cite{vector-store-index}. Ak nechceme nahrávať všetky dokumenty do~pamäte pri~využívani indexu, ale~len tie najpodobnejšie, budeme musieť využiť službu externého vektorového úložiska~\cite{external-vector-store-service}.

LlamaIndex poskytuje podporu pre~viac ako 20 rôznych vektorových úložisk, napr.~Chroma, DuckDB, Pinecone atď. Keďže nevyžadujeme žiadnu špeciálnu funkcionalitu, a~keďže autor nemá rozsiahle skúseností so~žiadnou z~možností, vyberáme si možnosť Chroma. Ide o~open-source vektorovú databázu~\cite{chroma2}, ktorá nám umožní vektorové vyhľadávanie a~uloženie dokumentov~\cite{chroma}. V~knižnici LlamaIndex sa s~databázou pracuje pomocou triedy \texttt{ChromaVectorStore}, ktorá zabezpečí, že počas dotazovania index využije ChromaDB na~vyhľadanie len určitý počet najpodobnejších uzlov~\cite{chroma-vector-store}.

\section{Komunikácia}

V~tejto kapitole si predstavíme komunikáciu medzi jednotlivými časťami systému.

\subsection{Polling}

Po~nahratí datasetu sa spustí jeho analýza, ktorá je časovo náročná. Podľa opisu návrhu UI z~časti~\ref{datasety-navrh} bude môcť používateľ sledovať stav datasetu na~stránke s~datasetmi. Aby bola aplikácia schopná zobraziť aktuálny stav datasetu, bude Frontend periodicky získavať informáciu o~jeho stave od~Aplikačného servera, ktorý informáciu čerpá z~databázy (viď obr.~\ref{polling}). Možné hodnoty stavu datasetu budú:
\begin{itemize}
\item Created~-- dataset je vytvorený, ale~ešte sa nezaradil do~radu na~analýzu
\item Queued~-- dataset je zaradený do~radu na~analýzu, ale~ešte sa nezačalo s~jeho analýzou
\item \textit{Processing}~-- dataset sa analyzuje
\item \textit{Processed}~-- dataset je analyzovaný
\item \textit{Failed}~-- analýza datasetu skončila s~chybou
\end{itemize}

\begin{figure}[H]\centering
\includegraphics[width=140mm]{img/polling}
\caption{Polling stavu datasetu.}
\label{polling}
\end{figure}



--- \textbf{POZOR!} Do konca kapitoly nasleduje starý text ---



\section{Backend}

Kľúčové oblasti, ktoré bude backendová časť systému riešiť sú:
\begin{enumerate}
\item Autentifikácia
\item Analýza datasetu, do~ktorej patrí:
\begin{itemize}
\item Data profiling
\item Promptovanie LLM
\end{itemize}
\item Práca s~databázou, napr.~na~vytvorenie alebo~odstránenie chatu
\end{enumerate}

Backendová časť aplikácie bude rozdelená na~dve časti:
\begin{itemize}
\item Aplikačný server~-- rieši prvý a tretí bod
\item LLM server~-- rieši druhý bod
\end{itemize}

Dôvod rozdelenia si vysvetlíme ďalej v tejto kapitole.

\section{Výber technológií pre aplikačný server}

Na~implementáciu aplikačného servera je vhodné použiť jazyk~C\# v~kombinácii s~frameworkmi ASP.NET Core a~Entity Framework Core. Ide o~štandardnú a často používanú voľbu v~oblasti vývoja moderných webových aplikácií~\cite{top-backend-languages}, najmä v~prípade potreby silnej typovej kontroly a~integrácie s~relačnými databázami. Táto kombinácia poskytuje výhody v~podobe silného typového systému, ktorý prispieva k~znižovaniu behových chýb, ako aj pohodlného ORM (Object Relational Mapping) prístupu k~databáze. Vďaka využitiu Entity Framework Core a~prístupu \textit{Code First} nie je potrebné manuálne vytvárať SQL schémy~-- databázové tabuľky sú generované automaticky na~základe definícií v~kóde.

Z~pohľadu databázovej vrstvy je vzhľadom na~zvolený .NET stack prirodzenou voľbou relačná databáza Microsoft SQL Server, ktorá sa v praxi často využíva spolu s vyššie uvedenými technológiami. Autor má s jazykmi C\# a~s~technológiami ASP.NET Core a~Entity Framework Core predchádzajúce skúsenosti, čo taktiež prispelo k~výberu tohto riešenia.

\section{Výsledná architektúra}

Pôvodným zámerom bolo vytvoriť aplikačný server s~využitím ASP.NET Core, pričom väčšina backendovej logiky mala byť implementovaná v jazyku C\#. V~prípadoch, keď by bolo potrebné interagovať s~veľkým jazykovým modelom (LLM), by C\# kód spúšťal Python skripty, ktoré by pomocou knižnice \textit{LlamaIndex} zabezpečovali komunikáciu s~LLM.

Počas vývoja sa však ukázalo, že prístup volania Python scriptov zo~C\# kódu nie je z~časových dôvodov vhodný pre~niektoré operácie, napr.~generovanie odpovedí v~rámci chatu. Problém spočíva v~tom, že pri~každom spustení Python skriptu dochádza k~opätovnému načítaniu a~inicializácii všetkých potrebných knižníc, čo spôsobuje výrazné spomalenie celej operácie.

Ako riešenie tohto problému bolo navrhnuté vytvorenie samostatného LLM servera, ktorý zabezpečí jednorazové načítanie a~inicializáciu všetkých potrebných knižníc pri~spustení. Vďaka tomu je možné následne obsluhovať jednotlivé požiadavky efektívne, bez zbytočného opakovania úvodných operácií.
