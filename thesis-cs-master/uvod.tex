\chapter{Úvod}
%\addcontentsline{toc}{chapter}{Úvod}

Existuje množstvo datasetov rôzneho obsahu, veľkosti a~kvality. Výber konkrétneho datasetu však automaticky nezaručuje, že sa v~ňom budeme vedieť orientovať, ani že ho dokážeme efektívne využiť. Môže to byť spôsobené napríklad neznalosťou domény, nedostatočnou dokumentáciou datasetu a~podobne. V~takýchto prípadoch by dátovým analytikom, prípadne iným užívateľom datasetu, výrazne pomohla existencia experta oboznámeného s~daným datasetom, ktorý by im vedel poskytnúť vysvetlenie a~asistenciu pri~posúdení vhodnosti jeho využitia na~konkrétnu úlohu. V~súčasnosti umelá inteligencia zaznamenáva výrazný rozmach a~nachádza uplatnenie v~rôznych oblastiach. V~tejto práci ju využijeme na~vytvorenie chatbota na~vysvetľovanie datasetov.

Prirodzene sa vyskytuje otázka, či by nebolo možné využiť existujúce nástroje, ako je napríklad ChatGPT. S~primeraným promptovaním by užívateľ mohol získať informácie o~datasete a~nechať si ho vysvetliť. Naša webová aplikácia po~nahraní datasetu, prípadne aj schémy, vykoná analýzu datasetu, ktorá zahŕňa dátové profilovanie. Výstupom tejto analýzy je \textit{data profiling report} (ďalej len report). Následne sa pomocou predpripravených promptov bude komunikovať s~veľkým jazykovým modelom (LLM) s~cieľom získať čo najviac informácií, ktoré sa využijú na~vytvorenie uceleného štruktúrovaného súhrnu informácií o~datasete~-- \textit{dataset knowledge} (ďalej len DK). DK bude prezentovaný užívateľovi a~slúži na~vysvetlenie štruktúry a~obsahu datasetu. Ak by mal užívateľ ďalšie otázky, môže sa opýtať asistenta, prípadne ho môže aj požiadať o~úpravu DK.

Pridanou hodnotou nášho riešenia je, že užívateľ sa nemusí zaoberať tvorbou vlastných promptov ani dátovým profilovaním, čím sa šetrí čas a~úsilie. Ďalej, v~prípade, že je dataset príliš rozsiahly na~to, aby sa zmestil do~kontextového okna jazykového modelu by asistent, ako napríklad ChatGPT, nemusel mať dostatok informácií na~adekvátne zodpovedanie otázok. Naša aplikácia však bude mať prístup ku~všetkým poskytnutým dátam~-- CSV súborom, reportom a~prípadne aj k~schéme, ak bola poskytnutá. Navyše, aplikácia vygeneruje DK, ktorá detailne vysvetlí daný dataset.

\section{Cieľ práce}

Cieľom tejto diplomovej práce je navrhnúť a~implementovať webovú aplikáciu určenú pre~odborníkov, ako sú dátoví analytici a~softvéroví inžinieri, nie pre~bežných používateľov, ktorá po~nahraní CSV~súborov s~voliteľnou schémou pomôže lepšie pochopiť ich obsah a~zároveň poskytne podporu pri~rozhodovaní, či je daný dataset vhodný na~úlohu zamýšľanú užívateľom.
