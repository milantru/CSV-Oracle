\chapter*{Úvod}
\addcontentsline{toc}{chapter}{Úvod}

\section{Cieľ práce}

Cieľom diplomovej práce je navrhnúť a implementovať systém pre softvérových inžinierov, ktorý po nahraní CSV datasetu pomôže jeho lepšiemu pochopeniu a zároveň poskytne podporu pri rozhodovaní, či je daný dataset vhodný na ďalšie použitie. Systém umožní zhodnotiť, či dataset obsahuje relevantné údaje a či ho možno použiť na splnenie zámeru softvérového inžiniera (napr. vytvorenie softvéru).

Priebeh (flow) programu:
\begin{itemize}
\item Užívateľ nahrá CSV dataset
\item Pustí sa naň data profiling
\item Z výsledku data profilingu si vezmeme potrebné info.
\item Použijeme predpripravené prompty spolu s výsledkom data profilngu na vympromptovanie ľubovoľného LLM a získame tak info o datasete
\item Info o datasete zobrazíme užívateľovi a umožníme mu pýtať sa ďalej na otázky ohľadom datasetu
\item Pri chatovaní sa môžu objaviť nové informácie o datasete, tie sa majú rozpoznať a zobraziť pri už existujúcich (môžu sa prepísať/zhrnúť, aby spolu dávali zmysel, a aby neboli napr. 2 rovnaké vety)
\end{itemize}

\subsection{Pridaná hodnota}

Prečo využivať OracleCSV, nestačí ChatGPT? Teoreticky áno, stačilo by keby užívateľ využil ChatGPT a dostal by podobný výsledok. Pridaná hodnota OracleCSV spočíta v tom, že v prvej fázy systém pomocou predpripravených promptov získa bohaté info o datasete. Užívateľ si ich nemusí vymýšľať sám a ani nemusí vedieť ako sa robí data profiling, čiže dochádza k úspore času i úsilia. Ďalej užívateľovi umožníme rozprávať sa o dasete, pričom si ńebude musieť robiť poznámky, pretože systém mu ich (dokonca zosuamrizované) spíše zaňho.

\section{High-value datasety}

Otvorené dáta sú XYZ, sú zdarma, môžeme využívať. Existujú datasety, ktoré boli označené, že majú "vysokú hodnotu", volajú sa high-value datasety. Asi využijeme práve nejaký high-value dataset. Plus niekedy teraz v roku 2024 myslím, že vychádza zákon, že štáty EU musia vytvárať také datasety.

\section{Predpoklad anglicky hovoriaceho užívateľa}

Ako zistíme jazyk užívateľa? Predstava: Vykonajú sa predvytvorené prompty, získané info sa zobrazí v angličtine? Jazyk vieme získať z inputu, konkrétne z "Additional info" alebo "User view". Ale "Additional info" nemusí byť v jazyku užívateľa (možno ani view?). Návrhy:
\begin{itemize}
\item Užívateľ ako input zadá jazyk (príp. sa vezme z jazyku stránky ak viacjazyčná, ale to asi nebude, lebo to neni náš cieľ)
\item Prvý sa ozve užívateľ a z toho vydedukujeme jeho jazyk (Ako by užívateľ začal? Čo ak z view sa dá zistiť užívateľova otázka/odpoveď)
\item Začneme defaultne v angličtine (teda aj zistené info promptami bude vypísané v angličtine) a jazyk zmeníme:
\begin{itemize}
\item ak si to užívateľ vyžiada
\item vždy keď detekujeme zmenu jazyka (príde mi čudné ak by to systém stále text prekladal, čo ak by kvôli neustálym prekladom vznikli chyby?)
\end{itemize}
\item Obmedzíme sa len na angličtinu (momentálne je v inštrukciách, že užívateľ hovorí anglicky)
\end{itemize}

    