\chapter{Úvod}
%\addcontentsline{toc}{chapter}{Úvod}

Existuje množstvo datasetov rôzneho obsahu, veľkosti a~kvality. Výber konkrétneho datasetu však automaticky nezaručuje, že sa v~ňom budeme vedieť orientovať, ani že ho dokážeme efektívne využiť. Môže to byť spôsobené napríklad neznalosťou domény, nedostatočnou dokumentáciou datasetu a~podobne. V~takýchto prípadoch by dátovým analytikom, prípadne iným užívateľom datasetu, výrazne pomohla existencia asistenta oboznámeného s~daným datasetom.

Asistentovi by bol poskytnutý dataset na~preštudovanie, prípadne aj dodatočné informácie, ako napríklad jeho schéma. Po~oboznámení sa s~dátami by asistent poskytol informácie vysvetľujúce jednotlivé časti datasetu v~štrukturovanej forme a~umožnil by dopýtanie sa na~možné nejasnosti. Tým by asistent poskytol vysvetlenie datasetu a~asistenciu pri~posúdení vhodnosti využitia datasetu na~konkrétnu úlohu.

V~súčasnosti umelá inteligencia (ďalej len AI) zaznamenáva výrazný rozmach a~nachádza uplatnenie v~rôznych oblastiach. Vzniká preto otázka, či by na účel vytvorenia spomínaného asistenta nebolo možné využiť práve AI, konkrétne veľké jazykové modely (ďalej len LLM).

\section{Limitácie LLM}
\label{limitacie-llm}

Výberom LLM pre~tvorbu asistenta sú spojené aj obmedzenia, ktorými tieto modely disponujú. Poznanie ich obmedzení môže napomôcť k~efektívnejšiemu a~vhodnejšiemu využívaniu LLM. Medzi vybrané obmedzenia patria~\cite{llm-limitations}:
\begin{itemize}
\item \textbf{Vymýšľanie si informácií, tzv.~halucinácie}~-- LLM majú tendenciu nepriznať nevedomosť odpovede a~namiesto toho vytvoriť odpoveď, ktorá pôsobí dôveryhodne, hoci nie je pravdivá. % RAG

\item \textbf{Obmedzené schopnosti uvažovania}~-- LLM neboli primárne navrhnuté na~riešenie matematických problémov a~zvyknú mať problém so~základnými matematickými úlohami alebo~inými logickými úlohami. % data profiling

\item \textbf{Obmedzená dlhodobá pamäť}~-- Pri~každom použití začína LLM odznova, teda nepamätá si predchádzajúce správy pokiaľ mu ich nepripomenieme. % chat history

\item \textbf{Obmedzené poznatky}~-- LLM sú trénované na~obmedzených dátach. To znamená, že ak nemajú prístup k~internetu alebo~inému zdroju informácií, nie sú schopné poskytnúť presné odpovede na~otázky, ktorých odpovede zahŕňajú informácie mimo ich trénovania. % RAG

\item \textbf{,,Prompt hacking``}~-- LLM môžu byť používateľmi, ktorí vedia šikovne manipulovať s~promptmi, oklamané a~manipulované k~akciám, ktoré im boli zakázané. % nebude sa riesit
\end{itemize}

\section{Navrhované riešenie}

Navrhovaným riešením je webová aplikácia, ktorá umožní používateľovi nahrať dataset v~podobe jedného alebo~viacerých CSV súborov, prípadne aj CSVW schémy. Po~nahraní dát vykoná analýzu datasetu, ktorá zahŕňa dátové profilovanie. Výstupom profilovania je \textit{data profiling report} (ďalej len report). Následne sa pomocou predpripravených promptov bude komunikovať s~LLM s~cieľom získať čo najviac informácií, ktoré sa využijú na~vytvorenie uceleného štruktúrovaného súhrnu informácií o~datasete~-- \textit{dataset knowledge} (ďalej len DK). DK bude prezentovaný používateľovi a~bude slúžiť na~vysvetlenie štruktúry a~obsahu datasetu. V~prípade ďalších otázok môže používateľ komunikovať s~asistentom v~chate zameranom na~konkrétny dataset, prípadne ak sa počas konverzácie objavia nové informácie, môže používateľ požiadať asistenta o~doplnenie alebo~úpravu DK.

\section{Alternatívne riešenie}

Alternatívym riešením môže byť využitie existujúceho nástroja, akým je napríklad \textit{ChatGPT}. Používateľ môže nástroju poskynúť celý dataset alebo~len jeho časť, prípadne ďalšie dáta v~podobe výsledkov dátového profilovania alebo~schémy. S~primeraným promptovaním užívateľ môže získať informácie o~datasete a~nechať si ho vysvetliť. Tento spôsob však zahŕňa nevýhody. Používateľ by si musel sám:
\begin{itemize}
\item vykonať dátové profilovanie
\item predať modelu dáta tak, aby sa vošli do~jeho kontextového okna alebo~ak sa používateľ rozhodne nahrať súbor, tak do~povoleného limitu veľkosti súboru
\item pripraviť primerané prompty
\item manuálne zadávať prompty asistentovi
\item manuálne vytvoriť z~odpovedí asistenta štrukturovaný súhrn informácií, udržiavať ho niekde vedľa a~prípadne aj nájsť spôsob jeho vizualizácie
\end{itemize}

\section{Cieľ práce}

Cieľom tejto diplomovej práce je navrhnúť a~implementovať prototyp riešenia vo~forme webovej aplikácie určenej pre~odborníkov, ako sú dátoví analytici a~softvéroví inžinieri, teda nie pre~bežných používateľov. Aplikácia po~nahraní CSV súborov s~voliteľnou schémou vygeneruje DK~-- štrukturovaný súhrn informácií vysvetľujúci štruktúru datasetu a~význam jeho jednotlivých častí. Používateľovi bude DK sprístupnená v~aplikácii a~zároveň mu bude umožnená interaktívna komunikácia s~asistentom ohľadom daného datasetu s~cieľom dovysvetlenia prípadných nejasností. V~prípade výskytu nových informácií počas konverzácie môže používateľ asistenta požiadať o~zapísanie týchto údajov do~DK alebo~všeobecne o~jeho úpravu. Týmto spôsobom aplikácia uľahčí pochopenie datasetu a~poskytne podporu pri~rozhodovaní o~jeho vhodnosti na~zamýšľanú úlohu, zároveň poskytne používateľovi možnosť budovať a~udržiavať zistené informácie o~datasete v~podobe DK. Súčasťou práce je aj realizácia experimentov na~overenie funkčnosti navrhnutého riešenia.
