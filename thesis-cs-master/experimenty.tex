\chapter{Experimenty}

Táto kapitola najprv predstavuje využité datasety, štruktúru experimentov, analýzu meraní a následne prezentáciu výsledkov.

\section{Využité datasety}

V~tejto práci budeme z~otvorenych dát využívať konkrétne datasety:
\begin{enumerate}
\item \href{https://data.gov.cz/datov\%C3\%A1-sada?iri=https\%3A\%2F\%2Fdata.gov.cz\%2Fzdroj\%2Fdatov\%C3\%A9-sady\%2F00024341\%2Fc5b85b8c662a72b9c0c13bea4a098448}{COVID-19 dataset}~-- Dátová sada pozostávajúca z~jedného CSV súboru obsahujúca rôzne typy meraní z~obdobia pandémie ochorenia COVID-19, ako napríklad denné počty nakazených, vyliečených osôb, úmrtí a~ďalšie súvisiace údaje.
\item \href{https://data.europa.eu/data/datasets/92437de0-a730-428a-921d-1d9c418072d6?locale=en}{Population structure 2011}~-- Táto dátová sada obsahuje rôzne štatistické údaje o~štruktúre obyvateľstva v~Hanseatic a~University City of Rostock.
\end{enumerate}

Prvý dataset pozostáva z~jedného CSV súboru, ku~ktorému je k dispozícii aj schéma. V~tomto prípade ide konkrétne o~schému vo~formáte CSVW~\cite{csvw} (CSV on the Web). CSVW je štandard určený na opis a štruktúrovanie obsahu CSV tabuliek, čím uľahčuje ich interpretáciu a~spracovanie.

Druhý dataset, ktorý patrí medzi tzv.~high-value datasets, pozostáva z~ viacerých súborov, z~ktorých v~rámci tejto práce budú využité nasledovné štyri:
\begin{itemize}
\item \texttt{bevoelkerungsstruktur\_2011\_haushaltsstruktur.csv} (Budget struc\-tu\-re 2011)
\item \texttt{bevoelkerungsstruktur\_2011\_alter.csv} (Population by age 2011)
\item \texttt{bevoelkerungsstruktur\_2011\_geschlecht.csv} (Po\-pu\-la\-ti\-on\newline by gen\-der 2011)
\item \texttt{bevoelkerungsstruktur\_2011\_staatsangehoerigkeit.csv} (Po\-pu\-la\-ti\-on\newline by na\-ti\-o\-na\-li\-ty 2011)
\end{itemize}

\section{Štruktúra experimentov}

Budeme teda pracovať s dvoma datasetmi:
\begin{enumerate}
\item COVID-19 dataset
\item Population structure 2011
\end{enumerate}

Pre prvý dataset máme k dispozícii schému. Druhý patrí medzi high-value datasety a~pozostáva z~viacerých CSV súborov, schému nemáme.

V experimentoch budeme testovať ako sa zmení kvalita odpovedí asistenta keď je prítomná schéma. Okrem toho, budeme takisto sledovať ako sa zmení kvalita odpovedí asistenta pri poskytnutí rôznych variant nástrojov:
\begin{itemize}
\item individuálne query tooly + sub question query engine tool, ktorý je nimi tvorený (varianta experimentu 1)
\item iba individuálne query tooly (varianta experimentu 2)
\item iba sub question query engine tool tvorený individuálnymi toolmi (varianta experimentu 3)
\end{itemize}

\texttt{QueryEngineTool} (ďalej len QET) je nástroj využívajúci \texttt{QueryEngine}. \texttt{QueryEngine} je koncept z knižnice LlamaIndex, ktorý umožní používateľovi pýtať sa na otázky nad poskytnutými dátami. Používateľ mu môže zadať otázku v prirozdenom jazyku a nástroj vráti odpoveď. V aplikácii sú vytvorené nástroje pre zdroj informácií: CSV súbory, data profiling reporty, dataset knowledge, prípadne aj schéma ak pre používateľom poskytnutá. Každý QET vieme využívať individuálne ale takisto z nich vieme vytvoriť \texttt{SubQuestionQueryEngineTool}.

\texttt{SubQuestionQueryEngineTool} (ďalej len SQQET) tool je tvorený viacerými QET. V aplikácii je tvorený všetkými QET, jeden pre každý zdroj spomenutý skôr v texte. SQQET Takisto dostane otázku, vytvorí jednu alebo viacero podotázok, ktoré predá jednému alebo viacerým QET, ktoré má k dispozícii. Tie mu vrátia odpovede a na základe nich vytvorí odpoveď na pôvodnú otázku.

Varianta experimentu 1 používa kombináciu QET a SQQET, ktorý je z nich vytvorený. Využitie akéhokoľvek query engine toolu, či uŽ samostatného QET alebo SQQET, trvá. Experimentami je snaha zistiť, či by sa zlepšila kvalita odpovede keby zmením tooly podľa opisu vyššie.

Celkovo budeme mať varianty experimentov:
\begin{enumerate}
\item COVID-19 dataset bez schémy, varianta experimentu 1
\item COVID-19 dataset bez schémy, varianta experimentu 2
\item COVID-19 dataset bez schémy, varianta experimentu 3
\item COVID-19 dataset so schémov, varianta experimentu 1
\item COVID-19 dataset so schémov, varianta experimentu 2
\item COVID-19 dataset so schémov, varianta experimentu 3
\item Population structure 2011, varianta experimentu 1
\item Population structure 2011, varianta experimentu 2
\item Population structure 2011, varianta experimentu 3
\end{enumerate}

\section{Priebeh experimentov}

Nahráme dataset (príp. aj schému). Akonáhle program dokončí analýzu a vytvorí DK, tak vďaka pevnej predom definovanej štruktúre DK môžeme skontrolovať vyplnenosť DK.

Ďalej pre každý dataset existuje predpripravená séria otázok a očakávaných odpovedí. Keďže pracujeme s dvoma datasetmi, budeme mať celkovo 2 série otázok. Každú odpoveď asistenta porovnáme s predpripravenou odpoveďou.

\subsection{Séria otázok č.~1}

Táto séria otázok sa využije pri datasete COVID-19. Pri vytváraní nového chatu bude špecifikovaný \textit{user view}: ,,I want to use this dataset to create a simple dashboard with charts that show how COVID-19 developed over time, e.g. cases, deaths, and recoveries.``

Predpripravené otázky a odpovede:
\begin{enumerate}
\item Is it possible to show total infections, deaths, and recoveries in my app using this dataset? Which columns contain that information?

\textit{Yes, it's possible. To show the total number of infections, use \textbf{\texttt{ku\-mu\-lo\-va\-ny\_po\-cet\_na\-ka\-ze\-nych}}. For total deaths, use \textbf{\texttt{kumulovany\_pocet\_umrti}}, and for total recoveries, use \textbf{\texttt{kumulovany\_pocet\_vylecenych}}. These columns contain cumulative values, so they reflect the total counts up to each date.}

\item Is it possible to calculate active COVID-19 cases from this dataset? If so, how?

\textit{Active cases can be calculated as: \newline\textbf{\texttt{active\_cases = ku\-mu\-lo\-va\-ny\_po\-cet\_na\-ka\-ze\-nych \newline- ku\-mu\-lo\-va\-ny\_po\-cet\_vy\-le\-ce\-nych \newline- ku\-mu\-lo\-va\-ny\_po\-cet\_u\-mr\-ti}}. This gives the total number of people still infected at a given point.}

\item Is it possible to detect spikes or anomalies in the daily case numbers using this dataset? If so, how?

\textit{Spikes can be detected by calculating the daily differences using \textbf{\texttt{pri\-rust\-ko\-vy\_po\-cet\_na\-ka\-ze\-nych}} and flagging days where the daily change exceeds a predefined threshold (e.g. 2x the rolling average of daily cases).}

\item Can I estimate the test positivity rate from this data? If so, how can I calculate it, and which columns should I use?

\textit{Yes, the test positivity rate can be calculated as: test positivity rate = (\textbf{\textit{pri\-rust\-ko\-vy\_po\-cet\_na\-ka\-ze\-nych}} / \textbf{\textit{pri\-rust\-ko\-vy\_po\-cet\_pro\-ve\-de\-nych\_testu}}) * 100 This gives the percentage of tests that resulted in positive cases.}

\item Can I estimate the fatality rate from this data? If so, how can I calculate it, and which columns should I use?

\textit{Yes, the fatality rate can be calculated as: fatality rate = (\textbf{\textit{ku\-mu\-lo\-va\-ny\_po\-cet\_u\-mr\-ti}} / \textbf{\textit{ku\-mu\-lo\-va\-ny\_po\-cet\_na\-ka\-ze\-nych}}) * 100 This gives the percentage of tests that resulted in positive cases.}

\item What column shows how many COVID-19 tests were done each day?

\textit{You should use the column \textbf{\texttt{prirustkovy\_pocet\_provedenych\_testu}} to see how many PCR tests were done each day. If you also want to include antigen tests, use \textbf{\texttt{prirustkovy\_pocet\_provedenych\_ag\_testu}}. To get the total number of tests per day, you can sum both columns.}

\item Can I use this dataset to analyze which weekdays have the most and fewest COVID-19 tests performed?

\textit{Yes. is is possible.}

\item Can you please add this last information to the dataset knowledge to the mentioned column?

\textit{Yes, the dataset knowledge has been updated. [It really is updated correctly with the mentioned information.]}
\end{enumerate}

\subsection{Séria otázok č.~2}

Táto séria otázok sa využije pri datasete Population structure 2011. Pri vytváraní nového chatu bude špecifikovaný \textit{user view}: ,,I want to use this dataset to create a simple dashboard with charts that show how COVID-19 developed over time, e.g. cases, deaths, and recoveries.``

\begin{enumerate}
\item Can I calculate the average age for each urban area using this dataset? If so, which columns could I use for that?

\textit{Yes, you can calculate the average age for each urban area. The relevant column for this calculation is \textbf{\texttt{durchschnittsalter}}, which provides the average age for each urban area.}

\item Is it possible to create a heatmap that shows areas with the highest and lowest youth and elderly population ratios? If so, which columns should I use?

\textit{Yes, you can. You would use the \textbf{\texttt{jugendquotient}} column for youth population ratio and the \textbf{\texttt{altenquotient}} column for elderly population ratio. These will help identify areas with the highest and lowest ratios.}

\item Is it possible to view the areas and their percentages of households with children? If so, which columns should I use?

\textit{Yes, you can. The relevant column is \textbf{\texttt{anteil\_mit\_kindern}}, which shows the percentage of households with children in each urban area.}

\item Is it possible to identify which urban areas have a higher concentration of younger populations (under 30)? Which columns should I use?

\textit{Yes, you can. The best columns to use are \textbf{\texttt{anzahl\_juenger\_3}}, \textbf{\texttt{anzahl\_3\_6}}, \textbf{\texttt{anzahl\_6\_15}}, and \textbf{\texttt{anzahl\_15\_25}} to identify the number of younger people in specific age ranges. You can then calculate the concentration of younger populations in each area.}

\item Is it possible to visualize areas with the percentages of men and women? If so, which columns should I use?

\textit{Yes, it is possible. You can use the columns \textbf{\texttt{anteil\_maennlich}} (percentage of male population) and \textbf{\texttt{anteil\_weiblich}} (percentage of female population) from the ,,Population by gender 2011`` table. You can also include the corresponding urban area from \textbf{\texttt{stadtbereich\_bezeichnung}}.}

\item Can I display the proportion of foreign vs. German residents by area? Which columns should I use?

\textit{Yes, you can use \textbf{\texttt{anteil\_deutsch}} (percentage of German residents) and \textbf{\texttt{anteil\_auslaendisch}} (percentage of foreign residents) from the ,,Spatial Population Movement 2011`` table.}

\item Can you add this last information to the dataset knowledge to the referenced columns?

\textit{Yes, the dataset knowledge has been updated. [It really is updated correctly with the mentioned information.]}
\end{enumerate}

\section{Hodnotenie}

Každú odpoveď LLM, už či pri analýze alebo pri chatovaní, ohodnotíme číslom:
\begin{itemize}
\item 0\-- odpoveď neexistuje alebo neobsahuje požadované informácie, nezhoduje sa s očakávaným výsledkom
\item 1\-- odpoveď sa zhoduje s očakávaným výsledkom do istej miery, ale nie je úplna, napr. očakávaná odpoveď obsahuje viacero častí a jedna z nich chýba
\item 2\-- odpoveď sa zhoduje s očakávaným výsledkom
\end{itemize}

Ako bolo v úvode spomenuté po dokončení analýzy vytvoríme percentuálne vyhodnotenie naplnenosti DK. Celkovo budeme mať teda výsledok naplenia DK, výsledok úspešnosti.

Dataset pozostáva z viacerých CSV súborov. Pri niektorých odpovediach je spomenutý nie len názov stĺpca, ale aj názov tabuľky. Ako budeme vyhodnocovať? Ak užívateľ nešpecifikuje, ze chce aj názov tabuľky, nemusí tam byť?
// TODO 0,1,2

OTÁZKA: Čo ak využije kumulatívny namiesto prírastkový? Asi tiež brať ako validné, alebo nie?


\section{Ďalšie experimenty}

Existuje mnoho ďalších smerov akými by sme mohli experimenty rozšíriť. Experimenty by sme mohli rozšíriť:

\begin{itemize}
\item viacerými rôznymi datasetmi líšiacimi sa obsahom, veľkosťou súboru, počtom súborov
\item viacerými kombináciami indexov, napr. využiť iba index nad CSV súbormi
\item rôznymi promptovacími technikami a zmenou promptov, či už system promptov alebo promptov pri analýze
\item viacerými LLM rôzneho typu alebo veľkosti
\item viacerými otázkami
\item atď.
\end{itemize}
