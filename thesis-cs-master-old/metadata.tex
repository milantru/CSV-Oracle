%%% Vyplňte prosím základní údaje o závěrečné práci (odstraňte \xxx{...}).
%%% Automaticky se pak vloží na všechna místa, kde jsou potřeba.

% Druh práce:
%	"bc" pro bakalářskou
%	"mgr" pro diplomovou
%	"phd" pro disertační
%	"rig" pro rigorozní
\def\ThesisType{mgr}

% Název práce v jazyce práce (přesně podle zadání)
\def\ThesisTitle{Asistent pro vysvětlování datových sad postavený s využitím velkého jazykového modelu}

% Název práce v angličtině
\def\ThesisTitleEN{A dataset explanation assistant built using a large language model}

% Jméno autora (vy)
\def\ThesisAuthor{Bc. Milan Truchan}

% Rok odevzdání
\def\YearSubmitted{2025}

% Název katedry nebo ústavu, kde byla práce oficiálně zadána
% (dle Organizační struktury MFF UK:
% https://www.mff.cuni.cz/cs/fakulta/organizacni-struktura,
% případně plný název pracoviště mimo MFF)
\def\Department{Katedra softwarového inženýrství}
\def\DepartmentEN{Department of Software Engineering}

% Jedná se o katedru (department) nebo o ústav (institute)?
\def\DeptType{Katedra}
\def\DeptTypeEN{Department}

% Vedoucí práce: Jméno a příjmení s~tituly
\def\Supervisor{doc. Mgr. Martin Nečaský, Ph.D.}

% Pracoviště vedoucího (opět dle Organizační struktury MFF)
\def\SupervisorsDepartment{Katedra softwarového inženýrství}
\def\SupervisorsDepartmentEN{Department of Software Engineering}

% Studijní program (kromě rigorozních prací)
\def\StudyProgramme{Informatika - Softwarové a datové inženýrství (N0613A140015)}

% Nepovinné poděkování (vedoucímu práce, konzultantovi, tomu, kdo
% vám nosil pizzu a vařil čaj apod.)
\def\Dedication{%
Rád by som sa poďakoval vedúcemu práce doc. Mgr. Martinovi Nečaskému, Ph.D. za jeho vedenie, rady, ochotu a čas, ktorý mi venoval. Taktiež by som sa chcel poďakovať svojej rodine, ktorá ma motivovala a podporovala počas vypracovania tejto práce, ale aj počas celej doby štúdia.
}

% Abstrakt (doporučený rozsah cca 80-200 slov; nejedná se o zadání práce)
\def\Abstract{%
Cieľom práce bolo navrhnúť a vytvoriť webovú aplikáciu pre dátových analytikov a ďalších užívateľov, ktorí pracujú s rôznymi datasetmi. Aplikácia má za cieľ pomôcť užívateľom porozumieť datasetu a zistiť, či je vhodný pre užívateľom zamýšlanú úlohu.

Aplikácia umožňuje nahrať dataset, ktorý môže pozostávať z jedného alebo viacerých CSV súborov. Po nahratí prebehne analýza, v ktorej sa využíva dátové profilovanie datasetu a veľký jazykový model. Jej výsledkom je ucelený súhrn zistených informácií o štruktúre a významu položiek datasetu\~-- dataset knowledge. Ten je neskôr prezentovaný užívateľovi. Užívateľ si môže pre každý analyzovaný dataset vytvoriť chat s asistentom, v ktorom sa môže prirodzeným jazykom dopytovať na detaily týkajúce sa datasetu a zamýšľanej úlohy, prípadne požiadať asistenta o úpravu dataset knowledge.
}

% Anglická verze abstraktu
\def\AbstractEN{%
The goal of this master thesis was to design and create a web application for data analysts and other users who work with various datasets. The application aims to help users understand the dataset and determine whether it is suitable for the user's intended task.

The application allows the user to upload a dataset, which can consist of one or more CSV files. After uploading, an analysis is performed, which uses data profiling of the dataset and a large language model. The result is a comprehensive summary of the information found about the structure and meaning of the dataset items\~-- dataset knowledge. This is later presented to the user. The user can create a chat with an assistant for each analyzed dataset, in which they can ask for details about the dataset and the intended task in natural language, or ask the assistant to edit the dataset knowledge.
}

% 3 až 5 klíčových slov (doporučeno) oddělených \sep
% Hodí se pro nalezení práce podle tématu.
\def\ThesisKeywords{%
analýza dat\sep data profiling\sep velký jazykový model
}

\def\ThesisKeywordsEN{
data analysis\sep data profiling\sep large language model
}

% Pokud některá z položek metadat obsahuje TeXové řídící sekvence, je potřeba
% dodat i verzi v obyčejném textu, která se objeví v metadatech formátu XMP
% zabudovaných do výstupního souboru PDF. Pokud si nejste jistí, prohlédněte si
% vygenerovaný soubor thesis.xmpdata.
\def\ThesisAuthorXMP{\ThesisAuthor}
\def\ThesisTitleXMP{\ThesisTitle}
\def\ThesisKeywordsXMP{\ThesisKeywords}
\def\AbstractXMP{\Abstract}

% Máte-li dlouhý abstrakt a nechceme se mu vejít na informační stranu,
% můžete použít toto nastavení ke zmenšení písma informační strany.
% (Uvažte nicméně zkrácení abstraktu, to je často lepší.)
\def\InfoPageFont{}
%\def\InfoPageFont{\small}  % odkomentujte pro zmenšení písma
